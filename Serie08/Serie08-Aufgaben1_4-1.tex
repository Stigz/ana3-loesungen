\documentclass[10pt]{article}\usepackage[]{graphicx}\usepackage[]{color}
% maxwidth is the original width if it is less than linewidth
% otherwise use linewidth (to make sure the graphics do not exceed the margin)
\makeatletter
\def\maxwidth{ %
  \ifdim\Gin@nat@width>\linewidth
    \linewidth
  \else
    \Gin@nat@width
  \fi
}
\makeatother

\definecolor{fgcolor}{rgb}{0.345, 0.345, 0.345}
\newcommand{\hlnum}[1]{\textcolor[rgb]{0.686,0.059,0.569}{#1}}%
\newcommand{\hlstr}[1]{\textcolor[rgb]{0.192,0.494,0.8}{#1}}%
\newcommand{\hlcom}[1]{\textcolor[rgb]{0.678,0.584,0.686}{\textit{#1}}}%
\newcommand{\hlopt}[1]{\textcolor[rgb]{0,0,0}{#1}}%
\newcommand{\hlstd}[1]{\textcolor[rgb]{0.345,0.345,0.345}{#1}}%
\newcommand{\hlkwa}[1]{\textcolor[rgb]{0.161,0.373,0.58}{\textbf{#1}}}%
\newcommand{\hlkwb}[1]{\textcolor[rgb]{0.69,0.353,0.396}{#1}}%
\newcommand{\hlkwc}[1]{\textcolor[rgb]{0.333,0.667,0.333}{#1}}%
\newcommand{\hlkwd}[1]{\textcolor[rgb]{0.737,0.353,0.396}{\textbf{#1}}}%
\let\hlipl\hlkwb

\usepackage{framed}
\makeatletter
\newenvironment{kframe}{%
 \def\at@end@of@kframe{}%
 \ifinner\ifhmode%
  \def\at@end@of@kframe{\end{minipage}}%
  \begin{minipage}{\columnwidth}%
 \fi\fi%
 \def\FrameCommand##1{\hskip\@totalleftmargin \hskip-\fboxsep
 \colorbox{shadecolor}{##1}\hskip-\fboxsep
     % There is no \\@totalrightmargin, so:
     \hskip-\linewidth \hskip-\@totalleftmargin \hskip\columnwidth}%
 \MakeFramed {\advance\hsize-\width
   \@totalleftmargin\z@ \linewidth\hsize
   \@setminipage}}%
 {\par\unskip\endMakeFramed%
 \at@end@of@kframe}
\makeatother

\definecolor{shadecolor}{rgb}{.97, .97, .97}
\definecolor{messagecolor}{rgb}{0, 0, 0}
\definecolor{warningcolor}{rgb}{1, 0, 1}
\definecolor{errorcolor}{rgb}{1, 0, 0}
\newenvironment{knitrout}{}{} % an empty environment to be redefined in TeX

\usepackage{alltt}

\title{Serie 8}
\date{}
\usepackage{setspace}
\setstretch{1}
\usepackage{parskip}

\usepackage[utf8]{inputenc}
\usepackage[T1]{fontenc}
\usepackage{ngerman}
\usepackage{hyphenat}
\usepackage{natbib}
\usepackage[normalem]{ulem}

\usepackage[a4paper, top=2cm, bottom=2cm, left=2cm, right=2cm]{geometry}

% \usepackage[small, explicit]{titlesec}
% \usepackage[normalem]{ulem}
% \titleformat{\paragraph}[runin]{\normalfont\normalsize\itshape}{\theparagraph}{1em}{#1}
% \titleformat{\subparagraph}[runin]{\normalfont\normalsize}{\thesubparagraph}{1em}{\uline{#1}}
% \titlespacing*{\subparagraph}{0pt}{1ex}{1em}


% \usepackage{marginnote}
% \renewcommand*{\marginfont}{\footnotesize}

\usepackage{tikz}
\usetikzlibrary{calc, intersections, arrows}

\usepackage{amsmath}
\usepackage{amsfonts}
\usepackage{amssymb}
\usepackage{amsthm}
\usepackage{mathtools}
\usepackage{mathrsfs}

% \usepackage{tgpagella}
% \usepackage[sc]{mathpazo}

\renewcommand*{\proofname}{Beweis}
\newtheorem*{lemma}{Lemma}


\author{}
% \date{\today}

% \usepackage{booktabs}
% \usepackage{fancyhdr}
% \usepackage{lastpage}
% \pagestyle{fancy}
% \fancyhf{}
% \makeatletter
% \let\runauthor\@author
% \let\runtitle\@title
% \makeatother
% \lhead{\runauthor}
% \rhead{S.~\thepage~von \pageref{LastPage}\\\today}
% \chead{\runtitle}

\usepackage[shortlabels]{enumitem}

% \usepackage{microtype}
             
\newcommand{\N}{\mathbb{N}}
\newcommand{\Z}{\mathbb{Z}}
\newcommand{\Q}{\mathbb{Q}}
\newcommand{\R}{\mathbb{R}}
\newcommand{\K}{\mathbb{K}}
\newcommand{\C}{\mathbb{C}}
\newcommand{\df}{\,\textrm{d}}
\newcommand{\pr}{\,\textrm{pr}}

% \newcommand*{\QED}[1][$\square$]{%
% \leavevmode\unskip\penalty9999 \hbox{}\nobreak\hfill
%     \quad\hbox{#1}%
% }
                                
\hyphenation{Äqui-valenz-re-la-tion Äqui-va-lenz-klasse Prim-faktor-zer-legung 
Prim-faktor-zer-legungen unter-schied-lich-en sur-jek-tiv Po-tenzen 
unter-schied-liche In-duk-tions-an-nahme Differential-gleichung}

\usepackage{hyperref}
\IfFileExists{upquote.sty}{\usepackage{upquote}}{}
\begin{document}

%%%%%%%%%%%%%%%%%%%%%%%%%%%%%%%%%%%%%%%%%%%%%%%%%%%%%%%%%%%%%%%%%%%%%%%%%%%%%%%%
\maketitle

\uline{Bemerkung:} Meine Lösung für 1(c) stimmte nicht ganz,
weshalb ich sie hier auslasse.

%%%%%%%%%%%%%%%%%%%%%%%%%%%%%%%%%%%%%%%%%%%%%%%%%%%%%%%%%%%%%%%%%%%
%%%%%%%%%%%%%%%%%%%%%%%%%%%%%%%%%%%%%%%%%%%%%%%%%%%%%%%%%%%%%%%%%%%

%%%%%%%%%%%%%%%%%%%%%%%%%%%%%%%%%%%%%%%%%%%%%%%%%%%%%%%%%%%%%%%%%%%
%%%%%%%%%%%%%%%%%%%%%%%%%%%%%%%%%%%%%%%%%%%%%%%%%%%%%%%%%%%%%%%%%%%
\section*{Exercise 1}
\begin{enumerate}[(a)]
\item Show that the Dirichlet function is both Lebesgue-measurable
and Lebesgue-integrable (i.e., $D \in \mathcal{L}_1([0,1])$)
but not Riemann-integrable.
  \begin{proof}
    \uline{$D$ is Lebesgue-measurable:}
    Let $U \subset \R$ be open.
    If $\{0,1\} \not \subset U$, then $f^{-1}(U) = \emptyset \in \mathfrak{B}$.
    If $\{0,1\} \subset U$, then $f^{-1}(U) = [0,1] \in \mathfrak{B}$.
    If $0 \notin U, 1 \in U$, then $f^{-1}(U) = [0,1] \cap \Q$, viz., a countable 
    union of closed sets, hence $f^{-1}(U) \in \mathfrak{B}$.
    If $0 \in U, 1 \not \in U$, then $f^{-1}(U) = ([0,1] \cap \Q)^c \cap [0,1]$, which is in $\mathfrak{B}$
    by virtue of its complement's being in $\mathfrak{B}$.
    
    \uline{$D$ is Lebesgue-integrable:}
    $D$ is non-zero in a countable number of points. 
    Countable sets have Lebesgue-measure zero.
    Measure zero sets don't affect Lebesgue integrals.
    So $\int_{[0,1]} D \df m = 0 < \infty$. Hence $D \in \mathcal{L}_1([0,1])$.
    
    \uline{$D$ isn't Riemann-integrable:}
    Any interval in a partition of $[0,1]$ contains both rational and irrational numbers,
    so the upper Darboux sums are always 1 and the lower ones are always 0.
  \end{proof}
  
\item Show that $D$ can be made Riemann-integrable by tweaking its definition on a measure zero set.
  \begin{proof}
  See (a): $D = [0]$. (Equivalence class of zero function.)
  \end{proof}
  
\item Is there any bounded function $f: [0,1] \to \R$ that is Lebesgue-measurable
but not Riemann-integrable and that can't be made Riemann-integrable by tweaking its
definition on a measure zero set?

\textit{Die charakteristische Funktion einer `fetten' Cantor-Menge würde funktionieren. Google `fat Cantor set' :)}
  % \textit{Example.} I didn't have time to flesh this out, but:
  % \begin{enumerate}[1.]
  %   \item Create a `fat' Cantor set $A$ with $m(A) = 1/2$.
  %   \item Note that $\chi_A$ is discontinuous at all $x \in A$.
  %   \item Hence, $\chi_A$ isn't Riemann-integrable.
  %         However, its Lebesgue integral is $1/2$.
  %   \item Then argue that $\chi_A$ isn't equivalent to 
  %         a Riemann-integrable function either. (I reckon this would be the iffiest step.)
  % \end{enumerate}
\end{enumerate}

\section*{Exercise 2}
Compute the following Lebesgue integrals:
\[
  (a) \lim_{n \to \infty} \int_0^n \left(1 - \frac{x}{n}\right)^n e^{x/2} \df x,~~~
  (b) \lim_{n \to \infty} \int_0^n \left(1 + \frac{x}{n}\right)^n e^{-2x} \df x.
\]

\paragraph{Solution}
Note that 
\[
 \lim_{n \to \infty} \int_0^n \left(1 - \frac{x}{n}\right)^n e^{x/2} \df x = \lim_{n \to \infty} \int_0^{\infty} \underbrace{\left(1 - \frac{x}{n}\right)^n e^{x/2} \chi_{[0, n]}(x)}_{=: f_n(x)} \df x
\]
and
\[
 \lim_{n \to \infty} \int_0^n \left(1 + \frac{x}{n}\right)^n e^{-2x} \df x = \lim_{n \to \infty} \int_0^{\infty} \underbrace{\left(1 + \frac{x}{n}\right)^n e^{-2x} \chi_{[0, n]}(x)}_{=: g_n(x)} \df x.
\]
We will claim that the sequences $(f_n(x))_{n \in \N}, (g_n(x))_{n \in \N}$
are both dominated by $h(x) := e^{-x/2}$ and then leverage the dominated convergence theorem.
But first:

\begin{lemma}
For all $x > - n$, $\left(1 + \frac{x}{n}\right)^n \leq \left(1 + \frac{x}{n+1}\right)^{n+1}$.
\end{lemma}
\begin{proof} Using Bernoulli's inequality, we obtain:
\begin{align*}
  \frac{\left(1 + \frac{x}{n+1}\right)^{n+1}}{\left(1 + \frac{x}{n}\right)^n}
  &= \left(\frac{1 + \frac{x}{n+1}}{1 + \frac{x}{n}}\right)^{n+1}\left(1 + \frac{x}{n}\right) \\
  &= \left(\frac{(n+1+x)n}{(n+x)(n+1)}\right)^{n+1}\left(1 + \frac{x}{n}\right) \\
  &= \left(1 - \frac{x}{(n+x)(n+1)}\right)^{n+1}\left(1 + \frac{x}{n}\right) \\
  &\geq \left(1 - \frac{x}{n+x} \right)\left(1 + \frac{x}{n}\right) \\
  &= 1 + \frac{(n+x)x - xn - x^2}{n(n+x)} = 1.\qedhere
\end{align*}
\end{proof}

\uline{Claim 1: $f_n(x) \leq e^{-x/2}$:}
For $x \in \R \setminus [0,n)$, $f_n(x) = 0$, whereas $e^{-x/2} > 0$ for all $x$.
For $x \in [0,n)$, observe that
\[
 \left(1 - \frac{x}{n}\right)^n e^{x/2} \stackrel{!}{\leq} e^{-x/2}
\]
is equivalent to
\[
  \left(1 - \frac{x}{n}\right)^n \stackrel{!}{\leq} e^{-x} = \lim_{k \to \infty}\left(1 - \frac{x}{k}\right)^k.
\]
Since $-x > -n$, the lemma applies, so $\left(1 - \frac{x}{n}\right)^n$ increases monotonically with $n$, which proves the claim.

\uline{Claim 2: $g_n(x) \leq e^{-x/2}$:}
For $x \in \R \setminus [0,n)$, $g_n(x) = 0$, whereas $e^{-x/2} > 0$ for all $x$.
For $x \in [0,n)$, observe that
\[
 \left(1 + \frac{x}{n}\right)^n e^{-2x} \stackrel{!}{\leq} e^{-x/2}
\]
is equivalent to
\[
  \left(1 + \frac{x}{n}\right)^n \stackrel{!}{\leq} e^{3x/2} = e^{x/2}\lim_{k \to \infty}\left(1 + \frac{x}{k}\right)^k.
\]
Since $x \geq 0 > - n$ and the lemma applies,
and since $e^{x/2} \geq 1$ for all $x \geq 0$,
the last inequality is indeed true.
Hence, $e^{-x/2}$ dominates $(g_n(x))_{n \in \N}$.

\uline{Claim 3: $\int_{0}^{\infty} e^{-x/2} \df x < \infty$:}
We note that $-2e^{-x/2}$ is the antiderivative of $e^{-x/2}$,
so 
\[
\int_{0}^{\infty} e^{-x/2} \df x = \lim_{t \to \infty}-2e^{-t/2} + 2 = 2 < \infty.
\]

\uline{Claim 4: $\lim_{n \to \infty} f_n(x), \lim_{n \to \infty} g_n(x)$ both exist for all $x \in \R$:}
Specifically,
\[
  f_n(x) \to 
  \begin{cases}
  0,                          & x < 0, \\
  e^{-x}e^{x/2} = e^{-x/2},   & x \geq 0,
  \end{cases}, ~~~n \to \infty.
\]
and
\[
  g_n(x) \to \begin{cases}
  0,                          & x < 0,\\
  e^{x}e^{-2} = e^{-x},       & x \geq 0,
  \end{cases}, ~~~n \to \infty.
\]

Hence we may apply the dominated convergence theorem, which gives us
\[
  \lim_{n \to \infty} \int_{0}^{\infty} f_n(x) \df x = \int_0^{\infty} e^{-x/2} \df x = 2,
\]
as already computed above.
Similarly,
\[
  \lim_{n \to \infty} \int_{0}^{\infty} g_n(x) \df x = \int_0^{\infty} e^{-x} \df x = \lim_{t \to \infty}-e^{-t} + 1 = 1.
\]


\section*{Aufgabe 3}
Es seien $f_n: [0,1] \to [0, \infty)$ stetige Funktionen
mit $f_n(x) \to 0, n \to \infty,$ für alle $x \in [0,1]$.
Zeige, dass für $\varphi: [0,1] \to [0, \infty], \varphi := \sup_{n \in \N} f_n$, gilt:
\begin{enumerate}[(a)]
  \item $\varphi$ Lebesgue-integrierbar $\Rightarrow$ $\lim_{n\to \infty} \int_0^1 f_n \df x = 0.$
  \begin{proof}
  Es gilt $\varphi \in \mathscr{L}_1([0,1])$, $\varphi \geq 0$, $|f_n(x)| \leq \varphi(x)$ für alle $x \in [0,1]$ und alle $n \in \N$. Weiter $\lim_{n \to \infty} f_n(x) = 0$ für alle $x \in [0,1]$. Mit Satz 3.4 (majorisierte Konvergenz) folgt:
  \[
    \lim_{n \to \infty} \int_{[0,1]} f_n \df m = \int_{[0,1]} 0 \df m = 0.
  \]
  Da $f_n$ für alle $n$ stetig und folglich Riemann-integrierbar ist, gilt
  weiter
  \[
    \int_{[0,1]} f_n \df m = \int_0^1 f_n(x) \df x
  \]
  für alle $n \in \N$ und mithin die Behauptung.
  \end{proof}
  
  \item Existiert eine Funktionenfolge $(f_n)_{n \in \N}$
  wie oben, sodass $\varphi$ nicht Lebesgue-integrierbar
  ist, aber dennoch $\lim_{n \to \infty} \int_0^1 f_n \df x = 0$
  gilt?
  
  \textit{Antwort.} Ja. Betrachte die Funktionenfolge
  $(f_n)_{n\in \N}$ auf $[0,1]$, 
  der Funktionsgraph von $f_n$ aussieht wie
  ein gleichschenkliges Dreieck mit der Höhe $h_n = 2(n+1)$
  und mit dem Interval  $\left[\frac{1}{n+1}, \frac{1}{n}\right]$
  als Basis (entsprechend: Breite $b_n = \frac{1}{n^2+n}$);
  anderswo gilt $f_n(x) = 0$.
  Diese $f_n$ sind stetig.
  
  Für jedes $f_n$ gilt dann:
  \[
    \int_0^1 f_n(x) \df x = \frac{h_n\cdot b_n}{2} = \frac{n+1}{n^2+n} = \frac{1}{n} \to 0, n \to \infty.
  \]
  
  Da $\left[\frac{1}{m+1}, \frac{1}{m}\right] \cap \left[\frac{1}{n+1}, \frac{1}{n}\right] = \{1/m\}$ für $m = n + 1$
  und $f(1/n) = 0$ für alle $n \in \N$,
  kann man zum Berechnen von $\int_{[0,1]} \varphi \df m$
  die Integrale der einzelnen $f_n$ addieren:
  \[
    \int_{[0,1]} \varphi \df m = \sum_{k = 1}^{n} \int_0^1 f_k(x) \df x = \sum_{k=1}^{n} \frac{1}{k} \to \infty, n \to \infty.
  \]
  Somit gilt $\varphi \notin \mathscr{L}_1([0,1])$.
\end{enumerate}

\section*{Aufgabe 4}
Berechne das Lebesgue-Mass
der Menge
$K := \{(x,y) \in \R^2 : x^2 + y^2 \leq 1\} \subset \R^2$.

\paragraph{Lösung}
Ich bin nicht dazu gekommen, es ganz auszuformulieren,
aber von der Idee her:
\begin{enumerate}[1.]
  \item $m(\{(x,y) \in \R^2 : x, y \geq 0, x^2 + y^2 \leq 1\})$ mit Rechtecken
        der Breite $1/n$ von unten und von oben approximieren, $n \to \infty$:
        
\begin{knitrout}
\definecolor{shadecolor}{rgb}{0.969, 0.969, 0.969}\color{fgcolor}
\includegraphics[width=5cm,height=5cm]{figure/unnamed-chunk-1-1} 
\end{knitrout}

  \item Feststellen, dass dies das Riemannintegral $\int_0^1 \sqrt{1 - x^2} \df x$ ergibt.
  \item Tun, als ob man $\sqrt{1 - x^2}$ aus dem Stegreif integrieren kann, und dabei $\pi/4$ als Lösung erhalten.
  \item Symmetrie/Translationsinvarianz: $\leadsto m(K) = \pi$.
\end{enumerate}


\end{document}
