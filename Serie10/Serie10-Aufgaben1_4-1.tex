\documentclass[10pt]{article}\usepackage[]{graphicx}\usepackage[]{color}
% maxwidth is the original width if it is less than linewidth
% otherwise use linewidth (to make sure the graphics do not exceed the margin)
\makeatletter
\def\maxwidth{ %
  \ifdim\Gin@nat@width>\linewidth
    \linewidth
  \else
    \Gin@nat@width
  \fi
}
\makeatother

\definecolor{fgcolor}{rgb}{0.345, 0.345, 0.345}
\newcommand{\hlnum}[1]{\textcolor[rgb]{0.686,0.059,0.569}{#1}}%
\newcommand{\hlstr}[1]{\textcolor[rgb]{0.192,0.494,0.8}{#1}}%
\newcommand{\hlcom}[1]{\textcolor[rgb]{0.678,0.584,0.686}{\textit{#1}}}%
\newcommand{\hlopt}[1]{\textcolor[rgb]{0,0,0}{#1}}%
\newcommand{\hlstd}[1]{\textcolor[rgb]{0.345,0.345,0.345}{#1}}%
\newcommand{\hlkwa}[1]{\textcolor[rgb]{0.161,0.373,0.58}{\textbf{#1}}}%
\newcommand{\hlkwb}[1]{\textcolor[rgb]{0.69,0.353,0.396}{#1}}%
\newcommand{\hlkwc}[1]{\textcolor[rgb]{0.333,0.667,0.333}{#1}}%
\newcommand{\hlkwd}[1]{\textcolor[rgb]{0.737,0.353,0.396}{\textbf{#1}}}%
\let\hlipl\hlkwb

\usepackage{framed}
\makeatletter
\newenvironment{kframe}{%
 \def\at@end@of@kframe{}%
 \ifinner\ifhmode%
  \def\at@end@of@kframe{\end{minipage}}%
  \begin{minipage}{\columnwidth}%
 \fi\fi%
 \def\FrameCommand##1{\hskip\@totalleftmargin \hskip-\fboxsep
 \colorbox{shadecolor}{##1}\hskip-\fboxsep
     % There is no \\@totalrightmargin, so:
     \hskip-\linewidth \hskip-\@totalleftmargin \hskip\columnwidth}%
 \MakeFramed {\advance\hsize-\width
   \@totalleftmargin\z@ \linewidth\hsize
   \@setminipage}}%
 {\par\unskip\endMakeFramed%
 \at@end@of@kframe}
\makeatother

\definecolor{shadecolor}{rgb}{.97, .97, .97}
\definecolor{messagecolor}{rgb}{0, 0, 0}
\definecolor{warningcolor}{rgb}{1, 0, 1}
\definecolor{errorcolor}{rgb}{1, 0, 0}
\newenvironment{knitrout}{}{} % an empty environment to be redefined in TeX

\usepackage{alltt}

\title{Serie 10}
\date{}
\usepackage{setspace}
\setstretch{1}
\usepackage{parskip}

\usepackage[utf8]{inputenc}
\usepackage[T1]{fontenc}
\usepackage{ngerman}
\usepackage{hyphenat}
\usepackage{natbib}
\usepackage[normalem]{ulem}

\usepackage[a4paper, top=2cm, bottom=2cm, left=2cm, right=2cm]{geometry}

% \usepackage[small, explicit]{titlesec}
% \usepackage[normalem]{ulem}
% \titleformat{\paragraph}[runin]{\normalfont\normalsize\itshape}{\theparagraph}{1em}{#1}
% \titleformat{\subparagraph}[runin]{\normalfont\normalsize}{\thesubparagraph}{1em}{\uline{#1}}
% \titlespacing*{\subparagraph}{0pt}{1ex}{1em}


% \usepackage{marginnote}
% \renewcommand*{\marginfont}{\footnotesize}

\usepackage{tikz}
\usetikzlibrary{calc, intersections, arrows}

\usepackage{amsmath}
\usepackage{amsfonts}
\usepackage{amssymb}
\usepackage{amsthm}
\usepackage{mathtools}
\usepackage{mathrsfs}

% \usepackage{tgpagella}
% \usepackage[sc]{mathpazo}

\renewcommand*{\proofname}{Beweis}
\newtheorem*{lemma}{Lemma}


\author{}
% \date{\today}

% \usepackage{booktabs}
% \usepackage{fancyhdr}
% \usepackage{lastpage}
% \pagestyle{fancy}
% \fancyhf{}
% \makeatletter
% \let\runauthor\@author
% \let\runtitle\@title
% \makeatother
% \lhead{\runauthor}
% \rhead{S.~\thepage~von \pageref{LastPage}\\\today}
% \chead{\runtitle}

\usepackage[shortlabels]{enumitem}

% \usepackage{microtype}
             
\newcommand{\N}{\mathbb{N}}
\newcommand{\Z}{\mathbb{Z}}
\newcommand{\Q}{\mathbb{Q}}
\newcommand{\R}{\mathbb{R}}
\newcommand{\K}{\mathbb{K}}
\newcommand{\C}{\mathbb{C}}
\newcommand{\df}{\,\textrm{d}}
\newcommand{\pr}{\,\textrm{pr}}
\DeclareMathOperator*{\esssup}{ess\,sup}
\DeclareMathOperator*{\supp}{supp}


% \newcommand*{\QED}[1][$\square$]{%
% \leavevmode\unskip\penalty9999 \hbox{}\nobreak\hfill
%     \quad\hbox{#1}%
% }
                                
\hyphenation{Äqui-valenz-re-la-tion Äqui-va-lenz-klasse Prim-faktor-zer-legung 
Prim-faktor-zer-legungen unter-schied-lich-en sur-jek-tiv Po-tenzen 
unter-schied-liche In-duk-tions-an-nahme Differential-gleichung}

\usepackage{hyperref}
\IfFileExists{upquote.sty}{\usepackage{upquote}}{}
\begin{document}

%%%%%%%%%%%%%%%%%%%%%%%%%%%%%%%%%%%%%%%%%%%%%%%%%%%%%%%%%%%%%%%%%%%%%%%%%%%%%%%%
\maketitle

%%%%%%%%%%%%%%%%%%%%%%%%%%%%%%%%%%%%%%%%%%%%%%%%%%%%
\section*{Aufgabe 1}
Es seien $X = Y = [0,1]$ und
$\mu = \nu$ das Lebesgue-Mass auf $[0,1]$.
Weiter seien $t_0 = 0, (t_n)_{n \in \N}$
eine monoton steigende Folge mit $t_n \to 1, n \to \infty$
und $(g_n)_{n \in \N}$ eine Folge messbarer Funktionen
mit $\supp g_n \subset (t_{n-1}, t_n)$
und $\int_{[0,1]} g_n \df \mu = 1, n \in \N$.
Zeige, dass
\begin{align*}
  f: X \times Y     &\to    \R,\\
  (x,y) &\mapsto \sum_{n=1}^{\infty}(g_n(x)-g_{n+1}(x))g_n(y),
\end{align*}
wohldefiniert ist und dass gilt
\begin{align*}
  \int_X \int_Y f(x,y) \df \nu \df \mu \neq \int_Y \int_X f(x,y) \df \mu \df \nu.
\end{align*}

\textbf{$f$ ist wohldefiniert.}
Für ein $x \in [0,1]$ existiert höchstens ein von $x$ abhängiges $n \in \N$,
sodass $g_n(x) \neq 0$.
Grund:
\begin{align*}
g_n(x) \neq 0 \Rightarrow x \in (t_{n-1}, t_n) \Rightarrow x \notin (t_{m-1}, t_m), m \neq n \Rightarrow g_m(x) = 0, m \neq n. 
\end{align*}
Für $(x,y) \in [0,1]^2$ gilt also
\[
  f(x,y) = \sum_{n=1}^{\infty}(g_n(x)-g_{n+1}(x))g_n(y) = (g_{k_y}(x) - g_{k_y+1}(x))g_{k_y}(y) < \infty
\]
für ein von $y$ abhängiges $k_y \in \N$. Also ist $f$ wohldefiniert.

\textbf{Integrationsreihenfolge.}
Für festes $y_0 \in [0,1]$ gilt
\[
  f(\cdot,y_0) = (g_{k}(\cdot) - g_{k+1}(\cdot))g_{k}(y_0),
\]
für ein $k \in \N$.
Also 
\begin{align*}
  \int_0^1 f(x, y_0) \df x
  &= \int_0^1 f(x, y_0) \df x \\
  &= \int_0^1 (g_{k}(x) - g_{k+1}(x))g_{k}(y_0) \df x \\
  &= g_k(y_0)\left(\int_0^1 g_k(x) \df x - \int_0^1 g_{k+1}(x) \df x \right) \\
  &= g_k(y_0)\left(1 - 1\right) = 0,
\end{align*}
da $\int_0^1 g_n(x) \df x = 1$ für alle $n \in \N$.
Also 
\[
  \int_Y \int_X f(x,y) \df x \df y = \int_Y 0 \df y = 0.
\]

Für festes $x_0 \in [0,1)$ (das Verhalten bei $x = 1$
wird das Integral nicht beeinflussen) ist die Summe im Integral endlich,
weshalb sie mit dem Integral vertauscht.
\begin{align*}
 \int_Y f(x,y) \df y
 &= \int_Y \sum_{n=1}^{\infty} (g_n(x_0)-g_{n+1}(x_0))g_n(y) \df y &\\
 &= \sum_{n=1}^{\infty} \int_Y (g_n(x_0)-g_{n+1}(x_0))g_n(y) \df y &\\
 &= \sum_{n=1}^{\infty} (g_n(x_0)-g_{n+1}(x_0)) & [\int_Y g_n(y) \df y = 1] \\
 &= g_1(x_0) - \underbrace{\lim_{n \to \infty} g_n(x_0)}_{= 0} & [\textrm{teleskopische Summe}] \\
 &= g_1(x_0). &
\end{align*}
Da $\int_X g_1(x) \df x = 1$, folgt
\[
  \int_X \int_Y f(x_y) \df y \df x = \int_X g_1(x) \df x = 1 \neq 0 =  \int_Y \int_X f(x,y) \df x.
\]

%%%%%%%%%%%%%%%%%%%%%%%%%%%%%%%%%%%%%%%%%%%%%%%%%%%%
\section*{Aufgabe 2}
Es seien $\|\cdot\|$ die euklidische Norm auf $\R^n$
und $K_n := \{x \in \R^n : \|x\| \leq 1\}$
die $n$-dimensionale Einheitskugel. Zeige:
\[
  \textrm{vol}(K_n) = 
  \begin{cases}
    \frac{2^{k+1}\pi^k}{1\cdot 3 \cdot \dots \cdot (2k+1)}, & n = 2k+1, k \in \N_0,\\
    \frac{\pi^k}{k!}, & n = 2k, k \in \N.
  \end{cases}
\]

\begin{proof}
Wir führen einen Induktionsbeweis.

\uline{Verankerung:}
Für $n = 1$
ist die Einheitskugel gleich dem Intervall $[-1,1]$.
Dessen Volumen beträgt 2, was mit der Aussage übereinstimmt.
Für $n = 2$ erhalten wir die Fläche des Einheitskreises, $\pi$.\footnote{Wir brauchen zwei Verankerungen, da wir 
beim Induktionsschritt die geraden und die ungeraden $n$ separat behandeln.}

\uline{Induktionsschritt:}
Betrachte nun $K_{n+1}$ für irgendein $n \in \N$. 
Für $K_{x_{n+1}}$ gilt
\begin{align*}
  K_{x_{n+1}} 
  &= \{(x_i)^{n}_{i=1} : (x_i)_{i=1}^{n+1} \in K_{n+1}\} \\
  &= \bigcup_{t \in [-1, 1]} K_{x_{n+1}, t},
\end{align*}
wo
\[
  K_{x_{n+1}, t} = \{(x_1, \dots, x_{n}) : \sum_{i=1}^{n}x_i^2 \leq 1 - t^2\}
\]
für $t \in [-1,1]$.
Für fixes $t \in [-1,1]$ ist $K_{x_{n+1},t}$ eine $n$-dimensionale Einheitskugel,
die um den Faktor $\sqrt{1-t^2}$ gestreckt wurde. Streckungen sind lineare
Abbildungen, und die Determinante dieser Streckung beträgt $(1-t^2)^{n/2}$.
Somit erhalten wir
\[
  \textrm{vol}(K_{x_{n+1}, t}) = (1-t^2)^{n/2}\textrm{vol}(K_n).
\]
Mit Cavalieri folgt
\[
  \textrm{vol}(K_{n+1}) = \textrm{vol}(K_{n}) \underbrace{\int_{-1}^1 (1-t^2)^{n/2} \df t}_{=: I_{n}}.
\]
Ohne Beweis: $I_n = \frac{n}{n+1}I_{n-2}$. Daraus
\[
I_nI_{n-1} = \frac{n}{n+1}I_{n-1}I_{n-2} = \frac{2(n!)}{(n+1)(n!)} I_1 I_0 = \frac{4}{n+1}I_1.
\]
Da $I_1 = \frac{\pi}{2}$, ist $I_nI_{n-1} = \frac{2\pi}{n+1}$. Daraus
\[
 \textrm{vol}(K_{n+1}) = I_n\textrm{vol}(K_{n}) = I_nI_{n-1}\textrm{vol}(K_{n-1}) = \frac{2\pi}{n+1}\textrm{vol}(K_{n-1}).
\]

Falls $n = 2k$, dann $n+2 = 2(k+1)$. Mit der Induktionsannahme folgt
\[
 \textrm{vol}(K_{n+2}) = \frac{2\pi}{n+2}\textrm{vol}(K_n) = \frac{\pi}{k+1} \cdot \frac{\pi^k}{k!} = \frac{\pi^{k+1}}{(k+1)!}. 
\]

Falls $n = 2k+1$, dann $n+2 = 2k+3$. Mit der Induktionsannahme folgt
\[
  \textrm{vol}(K_{n+2}) = \frac{2\pi}{n+2}\textrm{vol}(K_n) = \frac{2\pi}{2k+3} \cdot \frac{2^{k+1}\pi^k}{(1\cdot 3 \cdot \dots \cdot (2k+1))} =
  \frac{2^{k+2}\pi^{k+1}}{1 \cdot 3 \cdot \dots \cdot (2k+1)\cdot(2k+3)}.
\]
\end{proof}

%%%%%%%%%%%%%%%%%%%%%%%%%%%%%%%%%%%%%%%%%%%%%%%%%%%%
\section*{Aufgabe 3}
\begin{enumerate}[(a)]
  \item Berechne das Volumen des 3d-Körpers, der durch
        Rotation der Fläche
        \[
          \{(x, 0, z) \in \R^3 : x \in (0, 1], z \in [1, x^{-1}]\} \subset \R^3
        \]
        um die $z$-Achse entsteht.
        
  \textbf{Antwort.} Bezeichne den Körper, der so entsteht als $R'$.
  Die Nullmenge $\{(0, 0, z) : z \in [1, \infty)\}$ können wir $R'$
  noch hinzufügen, ohne sein Volumen zu ändern. Bezeiche das neue Objekt
  mit $R$. Wir stellen fest, dass $R$ eine Überlagerung von Scheiben
  mit Radius $z^{-1}$ ist, mit $z > 1$. Folglich
  \[
    \textrm{vol}(R) = \int_1^{\infty} \pi (z^{-1})^2 \df z = \pi \int_1^{\infty} z^{-2} = \pi.
  \]
        
  \item Berechne das Integral
  \[
    \int_0^1 \int_x^1 \underbrace{y^2\sin\left(\frac{2\pi x}{y}\right)}_{=: f(x,y)} \df y \df x.
  \]
  
  \textbf{Antwort.} Die Funktion $f$ wird beim Integrieren nur für 
  $(x,y)$ im offenen Dreieck aufgespannt von den Eckpunkten 
  $(0,0), (1, 0)$ und $(1, 1)$ ausgewertet. 
  Fubini greift (stetige Funktionen, endliches Intervall als Integrationsbereich)
  und wir können das Integral wie folgt umschreiben:
    \[
    \int_0^1 \int_x^1 y^2\sin\left(\frac{2\pi x}{y}\right) \df y \df x
    = \int_0^1 y^2 \int_0^y \sin\left(\frac{2\pi x}{y}\right) \df x \df y.
  \]
  Wir bemerken, dass 
  \[
  \frac{\df}{\df x} \left(-\frac{y}{2\pi} \cos\left(\frac{2\pi x}{y}\right)\right)
  = \sin\left(\frac{2\pi x}{y}\right).
  \]
  Also
  \[
    \int_0^y \sin\left(\frac{2\pi x}{y}\right)  = -\frac{y}{2\pi}(\cos(2\pi) - \cos(0)) = 0.
  \]
  Somit ist auch das gesuchte Integral gleich 0.
\end{enumerate}

%%%%%%%%%%%%%%%%%%%%%%%%%%%%%%%%%%%%%%%%%%%%%%%%%%%%%%%%%%%%%%%%%%%
\section*{Aufgabe 4}
Es sei $n \in \N$ und $\| \cdot \|$ die euklidische
Norm auf $\R^n$. Beweise, dass
\[
  \int_{\R^n} \exp(-\|x\|^2) \df x = \pi^{n/2}.
\]

\begin{proof}
  Wir setzen hier
  $\int_{0}^{\infty} \exp(-x^2) \df x = \frac{\sqrt{\pi}}{2}$ als bekannt voraus
  (oder \url{https://math.stackexchange.com/a/886561/830096}). 
  Aus Symmetriegründen folgt die Behauptung für $n = 1$.

  Per Induktion: Die Aussage gelte für $\R^n$. Betrachte nun $\R^{n+1}$
  mit $x = (x_1, \dots, x_n, x_{n+1}) \in \R^{n+1}$. Bemerke, dass
  $\exp(-\|x\|^2) > 0$ für alle $x$.
  Also greift der Satz von Fubini:
    \begin{align*}
   \int_{\R^{n+1}} \exp(-\|x\|^2) \df x
   &= \int_{\R}\left(\int_{\R^n} \exp(-(x_1^2+\dots+x_n^2) - x_{n+1}^2) \df (x_1, \dots x_n) \right) \df x_{n+1} \\
   &= \int_{\R}\left(\int_{\R^n} \exp(-(x_1^2+\dots+x_n^2)) \exp(-x_{n+1}^2) \df (x_1, \dots x_n) \right) \df x_{n+1} \\
   &= \int_{\R}\exp(-x_{n+1}^2)\underbrace{\left(\int_{\R^n} \exp(-\|(x_1, \dots, x_n)\|^2) \df (x_1, \dots, x_n)\right)}_{= \pi^{n/2}} \df x_{n+1} \\
   &= \pi^{n/2} \int_{\R}\exp(-x_{n+1}^2) \df x_{n+1} \\
   &= \pi^{n/2} \pi^{1/2} = \pi^{(n+1)/2}.\qedhere
  \end{align*}
\end{proof}
\end{document}
