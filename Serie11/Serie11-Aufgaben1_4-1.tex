\documentclass[10pt]{article}\usepackage[]{graphicx}\usepackage[]{color}
% maxwidth is the original width if it is less than linewidth
% otherwise use linewidth (to make sure the graphics do not exceed the margin)
\makeatletter
\def\maxwidth{ %
  \ifdim\Gin@nat@width>\linewidth
    \linewidth
  \else
    \Gin@nat@width
  \fi
}
\makeatother

\definecolor{fgcolor}{rgb}{0.345, 0.345, 0.345}
\newcommand{\hlnum}[1]{\textcolor[rgb]{0.686,0.059,0.569}{#1}}%
\newcommand{\hlstr}[1]{\textcolor[rgb]{0.192,0.494,0.8}{#1}}%
\newcommand{\hlcom}[1]{\textcolor[rgb]{0.678,0.584,0.686}{\textit{#1}}}%
\newcommand{\hlopt}[1]{\textcolor[rgb]{0,0,0}{#1}}%
\newcommand{\hlstd}[1]{\textcolor[rgb]{0.345,0.345,0.345}{#1}}%
\newcommand{\hlkwa}[1]{\textcolor[rgb]{0.161,0.373,0.58}{\textbf{#1}}}%
\newcommand{\hlkwb}[1]{\textcolor[rgb]{0.69,0.353,0.396}{#1}}%
\newcommand{\hlkwc}[1]{\textcolor[rgb]{0.333,0.667,0.333}{#1}}%
\newcommand{\hlkwd}[1]{\textcolor[rgb]{0.737,0.353,0.396}{\textbf{#1}}}%
\let\hlipl\hlkwb

\usepackage{framed}
\makeatletter
\newenvironment{kframe}{%
 \def\at@end@of@kframe{}%
 \ifinner\ifhmode%
  \def\at@end@of@kframe{\end{minipage}}%
  \begin{minipage}{\columnwidth}%
 \fi\fi%
 \def\FrameCommand##1{\hskip\@totalleftmargin \hskip-\fboxsep
 \colorbox{shadecolor}{##1}\hskip-\fboxsep
     % There is no \\@totalrightmargin, so:
     \hskip-\linewidth \hskip-\@totalleftmargin \hskip\columnwidth}%
 \MakeFramed {\advance\hsize-\width
   \@totalleftmargin\z@ \linewidth\hsize
   \@setminipage}}%
 {\par\unskip\endMakeFramed%
 \at@end@of@kframe}
\makeatother

\definecolor{shadecolor}{rgb}{.97, .97, .97}
\definecolor{messagecolor}{rgb}{0, 0, 0}
\definecolor{warningcolor}{rgb}{1, 0, 1}
\definecolor{errorcolor}{rgb}{1, 0, 0}
\newenvironment{knitrout}{}{} % an empty environment to be redefined in TeX

\usepackage{alltt}

\title{Serie 11}
\date{}
\usepackage{setspace}
\setstretch{1}
\usepackage{parskip}

\usepackage[utf8]{inputenc}
\usepackage[T1]{fontenc}
\usepackage{ngerman}
\usepackage{hyphenat}
\usepackage{natbib}
\usepackage[normalem]{ulem}

\usepackage[a4paper, top=2cm, bottom=2cm, left=2cm, right=2cm]{geometry}

% \usepackage[small, explicit]{titlesec}
% \usepackage[normalem]{ulem}
% \titleformat{\paragraph}[runin]{\normalfont\normalsize\itshape}{\theparagraph}{1em}{#1}
% \titleformat{\subparagraph}[runin]{\normalfont\normalsize}{\thesubparagraph}{1em}{\uline{#1}}
% \titlespacing*{\subparagraph}{0pt}{1ex}{1em}


% \usepackage{marginnote}
% \renewcommand*{\marginfont}{\footnotesize}

\usepackage{tikz}
\usetikzlibrary{calc, intersections, arrows}

\usepackage{amsmath}
\usepackage{amsfonts}
\usepackage{amssymb}
\usepackage{amsthm}
\usepackage{mathtools}
\usepackage{mathrsfs}

% \usepackage{tgpagella}
% \usepackage[sc]{mathpazo}

\renewcommand*{\proofname}{Beweis}
\newtheorem*{lemma}{Lemma}


\author{}
% \date{\today}

% \usepackage{booktabs}
% \usepackage{fancyhdr}
% \usepackage{lastpage}
% \pagestyle{fancy}
% \fancyhf{}
% \makeatletter
% \let\runauthor\@author
% \let\runtitle\@title
% \makeatother
% \lhead{\runauthor}
% \rhead{S.~\thepage~von \pageref{LastPage}\\\today}
% \chead{\runtitle}

\usepackage[shortlabels]{enumitem}

% \usepackage{microtype}
             
\newcommand{\N}{\mathbb{N}}
\newcommand{\Z}{\mathbb{Z}}
\newcommand{\Q}{\mathbb{Q}}
\newcommand{\R}{\mathbb{R}}
\newcommand{\K}{\mathbb{K}}
\newcommand{\C}{\mathbb{C}}
\newcommand{\df}{\,\textrm{d}}
\newcommand{\pr}{\,\textrm{pr}}
\DeclareMathOperator*{\esssup}{ess\,sup}
\DeclareMathOperator*{\supp}{supp}


% \newcommand*{\QED}[1][$\square$]{%
% \leavevmode\unskip\penalty9999 \hbox{}\nobreak\hfill
%     \quad\hbox{#1}%
% }
                                
\hyphenation{Äqui-valenz-re-la-tion Äqui-va-lenz-klasse Prim-faktor-zer-legung 
Prim-faktor-zer-legungen unter-schied-lich-en sur-jek-tiv Po-tenzen 
unter-schied-liche In-duk-tions-an-nahme Differential-gleichung}

\usepackage{hyperref}
\IfFileExists{upquote.sty}{\usepackage{upquote}}{}
\begin{document}

%%%%%%%%%%%%%%%%%%%%%%%%%%%%%%%%%%%%%%%%%%%%%%%%%%%%%%%%%%%%%%%%%%%%%%%%%%%%%%%%
\maketitle

\section*{Aufgabe 1}
\begin{enumerate}[(a)]
\item Zeige folgende Aussagen für die Fouriertransformationen
für $f,g \in \mathscr{L}_1(\R^n), \alpha \in \R^n$:
\begin{enumerate}[(i)]
  \item $(\tau_\alpha f)\widehat{} = e_{-\alpha}\widehat{f}$.
  \begin{proof}
      \begin{align*}
        (2\pi)^{n/2} (\tau_\alpha f)~\widehat{}
        &= \int_{\R^n} (\tau_\alpha f)(x) e^{-i\langle x, \cdot \rangle} \df x & [\textrm{Definition Fourier}]\\
        &= \int_{\R^n} f(x - \alpha) e^{-i\langle x, \cdot \rangle} \df x & [\textrm{Definition $\tau_\alpha$}]\\
        &= \int_{\R^n} f(u) e^{-i \langle u + \alpha, \cdot \rangle} \df u & [u = x - \alpha]\\
        &= \int_{\R^n} f(u) e^{-i \langle u, \cdot \rangle -i\langle \alpha , \cdot \rangle} \df u & [\textrm{Skalarprodukt bilinear}]\\
        &= e^{-i\langle \alpha, \cdot \rangle} \int_{\R^n} f(u) e^{-i \langle u, \cdot \rangle} \df u & \\
        &= e^{i\langle -\alpha, \cdot\rangle} \int_{\R^n} f(x) e^{-i \langle x, \cdot \rangle} \df x & [\textrm{Skalarprodukt bilinear}]\\
        &= (2\pi)^{n/2} e_{-\alpha}(\cdot)\widehat{f}. &  [\textrm{Definition Fourier, $e_{-\alpha}$}].
      \end{align*}
      \end{proof}
      
  \item $(e_\alpha f)\widehat{} = \tau_\alpha \widehat{f}$.
  \begin{proof}
\begin{align*}
  (2\pi)^{n/2} \widehat{e_\alpha f}
  &= \int_{\R^n} (e_\alpha f)(x) e^{-i\langle x, \cdot\rangle} \df x & [\textrm{Definition Fourier}] \\
  &= \int_{\R^n} f(x) e^{-i\langle x, \cdot\rangle + i\langle x, \alpha\rangle} \df x & [\textrm{Definition $e_\alpha(x)$}] \\
  &= \int_{\R^n} f(x) e^{-i\langle x, \cdot\rangle - i\langle x, -\alpha\rangle} \df x & [\textrm{Skalarprodukt bilinear}] \\
  &= \int_{\R^n} f(x) e^{-i\langle x, \cdot - \alpha\rangle} \df x &  \\
  &= (2\pi)^{n/2} \tau_{\alpha} \widehat{f}. & [\textrm{Definition Fourier, $\tau_\alpha$}]
\end{align*}
\end{proof}
  
  \item $\left( \overline{f^{\lor}}\right)\widehat{} = \overline{\widehat{f}}$.
  \begin{proof}
    Die komplexe Konjugation und das Integral vertauschen, also
    \begin{align*}
      (2\pi)^{-n/2} \overline{\widehat{f}}
      &= \overline{\int_{\R^n}f(x)e^{-i\langle x, \cdot \rangle} \df x} &\\
      &= \int_{\R^n}\overline{f(x)}\overline{e^{-i\langle x, \cdot \rangle} \df x} &\\
      &= \int_{\R^n}\left(\textrm{Re}(f(x)) - i\textrm{Im}(f(x))\right)e^{i\langle x, \cdot \rangle} \df x &\\
      &= \int_{\R^n}\left(\textrm{Re}(f(x)) - i\textrm{Im}(f(x))\right)e^{-i\langle -x, \cdot \rangle} \df x &\\
      &= \int_{\R^n}\left(\textrm{Re}(f(-x)) - i\textrm{Im}(f(-x))\right)e^{-i\langle x, \cdot \rangle} \df x & [T(x) = -x, |\det(T)| = 1] \\
      &= \int_{\R^n} \overline{f(-x)}e^{-i\langle x, \cdot \rangle} \df x \\
      &= (2\pi)^{-n/2}\widehat{\overline{f^{\lor}}}.& \qedhere
    \end{align*}
  \end{proof}
  
  \item $\lambda > 0, h(x) := f(\frac{x}{\lambda}), x \in \R^n \Rightarrow \widehat{h}(\xi) = \lambda^n \widehat{f}(\lambda \xi), \xi \in \R^n$.
  \begin{proof}
  Für $T(x) := x/\lambda$ gilt $|\det T| = \lambda^{-n}$.
  Somit:
\begin{align*}
  (2\pi)^{n/2} \widehat{h}(t)
  &= \int_{\R^n} h(x) e^{-i \langle x, t\rangle} \df x \\
  &= \int_{\R^n} f(T(x)) e^{-i \lambda \langle T(x), t \rangle} \df x \\
  &= \lambda^n \int_{\R^n} f(x) e^{-i \langle x, \lambda t \rangle} \df x \\
  &=  (2\pi)^{n/2} \lambda^n \widehat{f}(\lambda t).\qedhere
\end{align*}

  \end{proof}
  
  
  \item Es gilt $\widehat{f}g, f\widehat{g} \in \mathscr{L}_1(\R^n), \int_{\R^n}\widehat{f}g \df x = \int_{\R^n}f \widehat{g} \df x$.
  
  \begin{proof}
  $\widehat{f}, \widehat{g}$ sind beschränkt (Bemerkung 8.3) und stetig, also $\widehat{f}g, f\widehat{g} \in \mathscr{L}_1(\R^n)$. Mit Fubini:
\begin{align*}
  \int_{\R^n} f(x)\widehat{g}(x) \df x
  &= \int_{\R^n} f(x) \left((2\pi)^{-n/2} \int_{\R^n} g(y) e^{-i\langle y, x\rangle} \df y \right) \df x \\ 
  &= (2\pi)^{-n/2} \int_{\R^n} \int_{\R^n} g(y) f(x) e^{-i\langle y, x\rangle} \df y \df x \\
  &= (2\pi)^{-n/2} \int_{\R^n} g(y) \left(\int_{\R^n} f(x) e^{-i\langle x, y \rangle} \df x\right) \df y \\
  &= \int_{\R^n} g(y) \widehat{f}(y) \df y \\
  &=  \int_{\R^n} \widehat{f}(x) g(x)  \df x. \qedhere  \end{align*}
  \end{proof}
\end{enumerate}




\item Es seien $f \in \mathscr{L}_1(\R^n), y \in \R^n$ und $g: \R^n \to \R$,
      $g(y) = \|f-\tau_y f\|_1$. Zeige: $g(0) = 0$, $g \in \mathscr{L}_{\infty}(\R^n)$.
      
      \begin{proof}
        \uline{$g(0) = 0$:}
        \begin{align*}
          g(y)(x) &= \|f(x)-\tau_yf(x)\|_1 \\
          &= |f(x) - f(x-y)|.
        \end{align*}
        Also $g(0)(x) = 0$ für alle $x \in \R^n$.
        
        \uline{$g \in \mathscr{L}_{\infty}(\R^n)$:} (???)
      \end{proof}
\end{enumerate}


\section*{Aufgabe 2}
Berechne die Fouriertransformierten folgender Funktionen
und verifiziere an mindestens einem Beispiel die Ungleichung
$\| \widehat{f} \|_{\infty} \leq (2\pi)^{-n/2}\|f\|_1$:
\begin{enumerate}[(a)]
  \item $f: \R \to \R, f(x) = e^{-\lambda |x|},$ für eine Konstante $\lambda > 0$.
  
  
  \paragraph{Lösung.}
\begin{align*}
  \widehat{f}(t)
  &= \frac{1}{\sqrt{2\pi}} \int_{-\infty}^{\infty} e^{-\lambda|x|}e^{-ixt} \df x \\
  &= \frac{1}{\sqrt{2\pi}} \left(\int_{-\infty}^0 e^{\lambda x}e^{-ixt} \df x  + \int_{0}^{\infty} e^{-\lambda x}e^{-ixt} \df x\right) \\
  &= \frac{1}{\sqrt{2\pi}} \left(\int_{-\infty}^0 e^{\lambda x}(\cos(xt) - i\sin(xt)) \df x + \int_{0}^{\infty} e^{-\lambda x}(\cos(xt) - i\sin(xt)) \df x\right).
\end{align*}

\uline{$\int e^{\lambda x}\cos(xt) \df x (=: \star)$:} Berechnung durch partielle Integration
mit $f = \cos(xt), g' = e^{\lambda x}$, d.h.,
$f' = -t\sin(xt), g = \frac{1}{\lambda}e^{\lambda x}$.
Daraus:
\begin{align*}
  \star &= \frac{1}{\lambda} \left(\cos(xt)e^{\lambda x} + t\int \sin(tx)e^{\lambda x} \df x\right)
\end{align*}
Mit partieller Integration mit $f = \sin(tx), f' = t\cos(tx), g' = e^{\lambda x}, g = \frac{1}{\lambda}e^{\lambda x}$:
\begin{align*}
  \star &= \frac{1}{\lambda} \cos(xt)e^{\lambda x} + \frac{t}{\lambda^2} \sin(tx)e^{\lambda x} - \left(\frac{t}{\lambda}\right)^2 \star.
\end{align*}
Daraus:
\[
 \star = \frac{\lambda^2}{\lambda^2 + t^2}\left(\frac{1}{\lambda} \cos(xt)e^{\lambda x} + \frac{t}{\lambda^2}\sin(tx)e^{\lambda x}\right).
\]
Folglich:
\[
 \int_{-\infty}^0 e^{\lambda x}\cos(xt) \df x
 = \frac{\lambda}{\lambda^2 + t^2}.
\]

\uline{$\int e^{\lambda x}\sin(xt) \df x (=: \star)$:}
Berechnung mit partieller Integration mit
$f = \sin(xt), f' = t\cos(xt), g' = e^{\lambda x}, g = \frac{1}{\lambda}e^{\lambda x}$:
\begin{align*}
  \star &= \frac{1}{\lambda} \left(\sin(xt)e^{\lambda x} - t\int \cos(tx)e^{\lambda x} \df x\right).
\end{align*}
Jetzt mit
$f = \cos(xt), f' = -t\sin(xt),  g' = e^{\lambda x}, g = \frac{1}{\lambda}e^{\lambda x}$:
\begin{align*}
  \star &= 
  \frac{1}{\lambda}\sin(xt)e^{\lambda x} - \frac{t}{\lambda^2}\cos(xt)e^{\lambda x} - \left(\frac{t}{\lambda}\right)^2 \star.
\end{align*}
Daraus:
\[
  \star = \frac{\lambda^2}{\lambda^2 + t^2}\left(\frac{1}{\lambda}\sin(xt)e^{\lambda x} - \frac{t}{\lambda^2}\cos(xt)e^{\lambda x}\right).
\]
Folglich:
\[
 \int_{-\infty}^0 e^{\lambda x}\sin(xt) \df x
 = \frac{-t}{\lambda^2 + t^2}.
\]

\uline{Analog:}
\[
  \int_0^{\infty} e^{-\lambda x}\sin(xt) \df x = \frac{t}{\lambda^2 + t^2},
\]
\[
  \int_0^{\infty} e^{-\lambda}\cos(xt) \df x = \frac{\lambda}{\lambda^2 + t^2}.
\]

Die Gesamtlösung:
\begin{align*}
\widehat{f}(t)
&= \frac{1}{\sqrt{2\pi}(\lambda^2 + t^2)}(\lambda + it + \lambda - it) 
= \frac{2\lambda}{\sqrt{2\pi}(\lambda^2 + t^2)}.
\end{align*}
  
  \item $g : \R^2 \to \R, g(x_1, x_2) = \begin{cases} x_1x_2, & \textrm{falls $|x_1|, |x_2| \leq 1$,} \\ 0, & \textrm{sonst.} \end{cases}$

  \paragraph{Lösung (b).}
Mit Fubini:
\begin{align*}
 \widehat{g}(t_1, t_2)
 &= \frac{1}{2\pi} \int_{[-1, 1]^2} x_1x_2 e^{-ix_1t_1}e^{-ix_2t_2} \df x \\
 &= \frac{1}{2\pi} \int_{-1}^1 x_1 e^{-ix_1t_1} \left(\int_{-1}^{1} x_2 e^{-ix_2t_2} \df x_2 \right) \df x_1. \\
\end{align*}
Es gilt
\begin{align*}
\int_{-1}^{1} x_2 e^{-ix_2t_2} \df x_2
&= \int_{-1}^1 x_2(\cos(x_2t_2) - i\sin(x_2t_2)) \df x_2 \\
&= \int_{-1}^1 x_2\cos(x_2t_2) \df x_2 - i\int_{-1}^1 x_2 \sin(x_2t_2) \df x_2.
\end{align*}
Durch partielle Integration (mit Computer) findet man
\[
 \int_{-1}^1 x_2\cos(x_2t_2) \df x_2 = 0
\]
und
\[
 \int_{-1}^1 x_2\sin(x_2t_2) \df x_2 = \frac{2(\sin(t_2)-t_2\cos(t_2))}{t_2^2}.
\]
Also
\begin{align*}
  \widehat{g}(t_1, t_2)
  &= \frac{-2i(\sin(t_2)-t_2\cos(t_2))}{2\pi t_2^2} \int_{-1}^1 x_1 e^{-ix_1t_1} \df x_1. \\
  &= \frac{1}{2\pi}\left(\frac{-2i(\sin(t_2)-t_2\cos(t_2))}{t_2^2}\right)\left(\frac{-2i(\sin(t_1)-t_1\cos(t_1))}{t_1^2}\right) \\
  &= -\frac{2}{\pi(t_1t_2)^2}(\sin(t_1) - t_1\cos(t_1))(\sin(t_2)-t_2\cos(t_2)).
\end{align*}

  
\end{enumerate}

\section*{Aufgabe 3}
Zeige $\widehat{f} = f$ für $f: \R^n \to \R, f(x) := e^{-\|x\|^2/2}$.

\paragraph{Lösung.}
Für $x = (x_1, \dots, x_{n-1}, x_n) \in \R^n$,
definiere $\widetilde{x} = (x_1, \dots, x_{n-1}) \in \R^{n-1}$.
Definiere $\widetilde{t} \in \R^{n-1}$ analog für $t \in \R^n$.
Dann
\begin{align*}
  \widehat{f}(t)
  &= (2\pi)^{-n/2} \int_{\R^n} e^{-\|x\|^2/2} e^{-i\langle x, t \rangle} \df x \\
  &= (2\pi)^{-n/2} \int_{\R^n} e^{-\|\widetilde{x}\|^2/2}e^{-x_n^2/2} e^{-i\langle \widetilde{x}, \widetilde{t} \rangle}e^{-i\langle x_n, t_n \rangle} \df x \\
  &= (2\pi)^{-n/2} \int_{\R^{n-1}} e^{-\|\widetilde{x}\|^2/2}e^{-i\langle \widetilde{x}, \widetilde{t} \rangle} \left(\int_{\R} e^{-x_n^2/2} e^{-i x_n t_n} \df x_n \right) \df \widetilde{x}.
\end{align*}
Klügere Menschen können jetzt wohl zeigen, dass
\[
  \int_{\R} e^{-x_n^2/2} e^{-i x_n t_n} \df x_n = \sqrt{2\pi} e^{-x_n^2/2}.
\]
Induktiv würde dann
\[
 \widehat{f}(t) = (2\pi)^{-n/2} \sqrt{(2\pi)}^n e^{-x_1^2/2} \cdot \dots \cdot e^{-x_n^2/2} = e^{-\|x\|^2/2}
\]
folgen.

\section*{Aufgabe 4}
Es seien $1 \leq p < \infty, f \in \mathscr{L}_p(\R^n)$
und $\{\varphi_\varepsilon: \varepsilon > 0\}$
ein glättender Kern mit $\varphi \in C_c^{\infty}(\R^n)$.
Zeige:
\begin{enumerate}[(a)]
  \item $\varphi(\cdot - y) \cdot f(\cdot) \in \mathscr{L}_p(\R^n), y \in \R^n$.


  \item $\varphi * f \in C^{\infty}(\R^n)$.

  
  \item $\supp(\varphi * f) \subset \supp(f) + \supp(\varphi)$.

  \item Falls $f \in C_c(\R^n)$, dann konvergiert $\varphi_\varepsilon * f$ gleichmässig gegen $f$ für $\varepsilon \to 0$.

\end{enumerate}

(???)
\end{document}
