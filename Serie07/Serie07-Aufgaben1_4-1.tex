\documentclass[10pt]{article}\usepackage[]{graphicx}\usepackage[]{color}
% maxwidth is the original width if it is less than linewidth
% otherwise use linewidth (to make sure the graphics do not exceed the margin)
\makeatletter
\def\maxwidth{ %
  \ifdim\Gin@nat@width>\linewidth
    \linewidth
  \else
    \Gin@nat@width
  \fi
}
\makeatother

\definecolor{fgcolor}{rgb}{0.345, 0.345, 0.345}
\newcommand{\hlnum}[1]{\textcolor[rgb]{0.686,0.059,0.569}{#1}}%
\newcommand{\hlstr}[1]{\textcolor[rgb]{0.192,0.494,0.8}{#1}}%
\newcommand{\hlcom}[1]{\textcolor[rgb]{0.678,0.584,0.686}{\textit{#1}}}%
\newcommand{\hlopt}[1]{\textcolor[rgb]{0,0,0}{#1}}%
\newcommand{\hlstd}[1]{\textcolor[rgb]{0.345,0.345,0.345}{#1}}%
\newcommand{\hlkwa}[1]{\textcolor[rgb]{0.161,0.373,0.58}{\textbf{#1}}}%
\newcommand{\hlkwb}[1]{\textcolor[rgb]{0.69,0.353,0.396}{#1}}%
\newcommand{\hlkwc}[1]{\textcolor[rgb]{0.333,0.667,0.333}{#1}}%
\newcommand{\hlkwd}[1]{\textcolor[rgb]{0.737,0.353,0.396}{\textbf{#1}}}%
\let\hlipl\hlkwb

\usepackage{framed}
\makeatletter
\newenvironment{kframe}{%
 \def\at@end@of@kframe{}%
 \ifinner\ifhmode%
  \def\at@end@of@kframe{\end{minipage}}%
  \begin{minipage}{\columnwidth}%
 \fi\fi%
 \def\FrameCommand##1{\hskip\@totalleftmargin \hskip-\fboxsep
 \colorbox{shadecolor}{##1}\hskip-\fboxsep
     % There is no \\@totalrightmargin, so:
     \hskip-\linewidth \hskip-\@totalleftmargin \hskip\columnwidth}%
 \MakeFramed {\advance\hsize-\width
   \@totalleftmargin\z@ \linewidth\hsize
   \@setminipage}}%
 {\par\unskip\endMakeFramed%
 \at@end@of@kframe}
\makeatother

\definecolor{shadecolor}{rgb}{.97, .97, .97}
\definecolor{messagecolor}{rgb}{0, 0, 0}
\definecolor{warningcolor}{rgb}{1, 0, 1}
\definecolor{errorcolor}{rgb}{1, 0, 0}
\newenvironment{knitrout}{}{} % an empty environment to be redefined in TeX

\usepackage{alltt}

\title{Serie 7}
\date{}
\usepackage{setspace}
\setstretch{1}
\usepackage{parskip}

\usepackage[utf8]{inputenc}
\usepackage[T1]{fontenc}
\usepackage{ngerman}
\usepackage{hyphenat}
\usepackage{natbib}
\usepackage[normalem]{ulem}

\usepackage[a4paper, top=2cm, bottom=2cm, left=2cm, right=2cm]{geometry}

% \usepackage[small, explicit]{titlesec}
% \usepackage[normalem]{ulem}
% \titleformat{\paragraph}[runin]{\normalfont\normalsize\itshape}{\theparagraph}{1em}{#1}
% \titleformat{\subparagraph}[runin]{\normalfont\normalsize}{\thesubparagraph}{1em}{\uline{#1}}
% \titlespacing*{\subparagraph}{0pt}{1ex}{1em}


% \usepackage{marginnote}
% \renewcommand*{\marginfont}{\footnotesize}

\usepackage{tikz}
\usetikzlibrary{calc, intersections, arrows}

\usepackage{amsmath}
\usepackage{amsfonts}
\usepackage{amssymb}
\usepackage{amsthm}
\usepackage{mathtools}
\usepackage{mathrsfs}

% \usepackage{tgpagella}
% \usepackage[sc]{mathpazo}

\renewcommand*{\proofname}{Beweis}
\newtheorem*{lemma}{Lemma}


\author{}
% \date{\today}

% \usepackage{booktabs}
% \usepackage{fancyhdr}
% \usepackage{lastpage}
% \pagestyle{fancy}
% \fancyhf{}
% \makeatletter
% \let\runauthor\@author
% \let\runtitle\@title
% \makeatother
% \lhead{\runauthor}
% \rhead{S.~\thepage~von \pageref{LastPage}\\\today}
% \chead{\runtitle}

\usepackage[shortlabels]{enumitem}

% \usepackage{microtype}
             
\newcommand{\N}{\mathbb{N}}
\newcommand{\Z}{\mathbb{Z}}
\newcommand{\Q}{\mathbb{Q}}
\newcommand{\R}{\mathbb{R}}
\newcommand{\K}{\mathbb{K}}
\newcommand{\C}{\mathbb{C}}
\newcommand{\df}{\,\textrm{d}}
\newcommand{\pr}{\,\textrm{pr}}

% \newcommand*{\QED}[1][$\square$]{%
% \leavevmode\unskip\penalty9999 \hbox{}\nobreak\hfill
%     \quad\hbox{#1}%
% }
                                
\hyphenation{Äqui-valenz-re-la-tion Äqui-va-lenz-klasse Prim-faktor-zer-legung 
Prim-faktor-zer-legungen unter-schied-lich-en sur-jek-tiv Po-tenzen 
unter-schied-liche In-duk-tions-an-nahme Differential-gleichung}

\usepackage{hyperref}
\IfFileExists{upquote.sty}{\usepackage{upquote}}{}
\begin{document}

%%%%%%%%%%%%%%%%%%%%%%%%%%%%%%%%%%%%%%%%%%%%%%%%%%%%%%%%%%%%%%%%%%%%%%%%%%%%%%%%
\maketitle

\uline{Bemerkung:} Meine Lösung für 1(b) stimmte nicht ganz,
weshalb ich sie hier auslasse.

%%%%%%%%%%%%%%%%%%%%%%%%%%%%%%%%%%%%%%%%%%%%%%%%%%%%%%%%%%%%%%%%%%%
%%%%%%%%%%%%%%%%%%%%%%%%%%%%%%%%%%%%%%%%%%%%%%%%%%%%%%%%%%%%%%%%%%%

%%%%%%%%%%%%%%%%%%%%%%%%%%%%%%%%%%%%%%%%%%%%%%%%%%%%%%%%%%%%%%%%%%%
\section*{Aufgabe 1}
Es seien $(X, \mathfrak{M}, \mu)$ ein Massraum
und $f, f_n, n \in \N$ messbare Abbildungen $X \to \R$.
Man sagt, \textit{$(f_n)_{n \in \N}$ konvergiert im Mass}
gegen $f$, wenn für alle $\delta > 0$
\[
  \lim_{n \to \infty} \mu(\{x \in X : |f_n(x) - f(x)| 
    \geq \delta \}) = 0.
\]
Beweise:
\begin{enumerate}[(a)]
  \item Falls $(f_n)_{n \in \N}$ gegen $f$ im Mass konvergiert,
        dann gibt es eine Teilfolge von $(f_n)_{n \in \N}$,
        die fast überall gegen $f$ konvergiert.
  \begin{proof}
    $(f_n)_{n\in \N}$ konvergiert im Mass gegen $f$.
    Also existiert für alle $k \in \N_{\geq 1}$
    ein $n_k \in \N$, sodass für alle $n \geq n_k$ gilt: 
    \[
      \mu\bigg(\bigg\{x \in X : \underbrace{|f_n(x) - f(x)|}_{=: g_n(x)} \geq \frac{1}{k}\bigg\}\bigg) < \frac{1}{2^k}.
    \]
    Folglich finden wir $n_1 < n_2 < n_3 < \dots$, sodass
    \[
      \mu\bigg(\underbrace{\bigg\{x \in X : g_{n_k}(x) \geq \frac{1}{k}\bigg\}}_{=: A_k}\bigg) < \frac{1}{2^k}.
    \]
    Definiere
    \[
      \bigcap_{\ell \geq 1}\bigcup_{k \geq \ell} A_k =: A.
    \]
    Dann
    \begin{align*}
      \mu(A) 
      &= \mu\left(\lim_{\ell \to \infty} \bigcup_{k \geq \ell} A_k\right)
      = \lim_{\ell \to \infty} \mu\left(\bigcup_{k \geq \ell} A_k\right) \\
      &\leq \lim_{\ell \to \infty} \sum_{k = \ell}^{\infty} \mu(A_k)
      \leq \lim_{\ell \to \infty} \frac{1}{2^{\ell}}\underbrace{\sum_{j = 0}^{\infty} \frac{1}{2^j}}_{= 2} \\
      &= \lim_{\ell \to \infty}  \frac{1}{2^{\ell - 1}}
      = 0.
    \end{align*}
    
    $x \in A$ heisst: $\forall \ell \in \N_{\geq 1} \exists k \geq \ell: g_{n_k}(x) \geq 1/k$.
    
    Also heisst $x \in A^c$: $\exists \ell \in \N_{\geq 1} \forall k \geq \ell: g_{n_k}(x) < 1/k$.
    
    Also gilt für alle $x \in A^c$: $g_{n_k}(x) \to 0, k \to \infty$.
    
    Also $f_{n_k} \to f$ fast überall, $k \to \infty$.
  \end{proof}

  \item Falls $\mu(X) < \infty$, dann sind äquivalent:
  \begin{enumerate}[(i)]
    \item $(f_n)_{n \in \N}$ konvergiert im Mass gegen $f$.

    \item Jede Teilfolge von $(f_n)_{n \in \N}$ hat eine
          Teilfolge, die fast überall gegen $f$ konvergiert.
  \end{enumerate}

  \textit{Lösung fehlt.}
  % \begin{proof}
  %   \uline{$(i) \Rightarrow (ii)$:}
  %   Wenn $(f_n)_{n \in \N}$ im Mass gegen $f$ konvergiert,
  %   dann auch jede Teilfolge $(f_{n_k})_{k \in \N}$. 
  %   Aus Teil (a) folgt, dass $(f_{n_k})_{k \in \N}$ eine
  %   Teilfolge hat, die fast überall gegen $f$ konvergiert.
  %   
  %   \uline{$(ii) \Rightarrow (i)$:}
  %   Nimm an, $(f_n)$ konvergiert nicht im Mass gegen $f$.
  %   Dann existiert eine Teilfolge $(f_{n_k}) \subset (f_n)$,
  %   sodass es $\delta, \varepsilon > 0$ gibt und für alle $n_k \in \N$
  %   \[
  %     \mu\big(\{|f_{n_k}(x) - f(x)| \geq \delta\}\big) > \varepsilon
  %   \]
  %   gilt.
  %   Nach Annahme existiert eine Teil-Teilfolge
  %   $(f_{n_{k_{\ell}}}) \subset (f_{n_k})$, die fast überall
  %   gegen $f$ konvergiert.
  %   
  %   Im Falle $\mu(X) < \infty$ folgt,\marginnote{Ich verwende
  %   hierbei den folgenden Hilfsatz, dessen Beweis mir entgeht:
  %   \begin{lemma}
  %   Falls $\mu(X) < \infty$ und $f_n \to f$ fast überall, $n \to \infty$, dann $f_n \to f$ im Mass, $n \to \infty$.
  %   \end{lemma}} dass
  %   $(f_{n_{k_{\ell}}}) \to f$ im Mass, $\ell \to \infty$.
  %   Also existieren doch $n_k \in \N$,
  %   für die
  %   \[
  %      \mu\big(\{|f_{n_k}(x) - f(x)| \geq \delta\}\big) \leq \varepsilon
  %   \]
  %   gilt. Widerspruch.
  % \end{proof}
  % 
  
  Die Äquivalenz gilt nicht, falls $\mu(X) < \infty$
  nicht vorausgesetzt ist.

  \paragraph{Beispiel.} Definiere für alle $n \in \N$
  \begin{align*}
    f_n : \R &\to \R,\\
          x &\mapsto \begin{cases}
            1, &x > n, \\
            0, &x \leq n.
          \end{cases}
  \end{align*}
  Die Funktionenfolge konvergiert punktweise 
  gegen die Nullfunktion. Aber für jedes Folgenglied
  gilt mit $\delta = 1/2$:
  \[
    m(\{x \in \R : |f_n(x)| \geq 1/2\}) = m((n, \infty)) = \infty.
  \]
  Also konvergiert die Folge im Mass nicht zur Nullfunktion. 
\end{enumerate}

%%%%%%%%%%%%%%%%%%%%%%%%%%%%%%%%%%%%%%%%%%%%%%%%%%%%%%%%%%%%%
\section*{Aufgabe 2}
Es sei $\alpha \in \R$.
Zeige, dass folgende Funktion Lebesgue-messbar ist:
\begin{align*}
  f:    (-1, 1)   &\to        \R,\\
        x         &\mapsto
        \begin{cases}
          x^{-1},  & x \in (-1,1)\setminus\{0\},\\
          \alpha,  & x = 0.
        \end{cases}
\end{align*}
\begin{proof}
  \uline{Fall 1:} Sei $U \subset \R$ offen mit $\alpha \notin U$.
  Dann ist $f^{-1}(U)$ offen, da $f$ auf $(-1,1) \setminus \{0\}$
  stetig ist.
  Dann $f^{-1}(U) \in \mathfrak{B}$.
  Also ist $f^{-1}(U)$ eine Lebesgue-messbare Menge.
  
  \uline{Fall 2:} Sei $\alpha \in U \subset \R$, $U$ offen.
  Dann ist auch $U \setminus \{\alpha\} = U \cap (\R\setminus\{\alpha\})$ offen.
  Also
  \[
    f^{-1}(U) = \underbrace{f^{-1}(U \setminus \{\alpha\})}_{\textrm{offen}} \sqcup \underbrace{f^{-1}(\{\alpha\}).}_{\substack{\textrm{höchstens zwei Punkte,}\\\textrm{also abgeschlossen}}}
  \]
  Offene und abgeschlossene Mengen sind Borelmengen. Ihre Vereinigung auch.
  Also ist $f^{-1}(U)$ eine Lebesgue-messbar Menge.
  
  Also ist $f$ Lebesgue-messbar.
\end{proof}

Ist $f$ Lebesgue-integrierbar?

\paragraph{Antwort.} Für $[a,b] \subset (-1,1)$ stimmen
das Lebesgue- und das Riemann-Integral überein.
Da $\{0\}, \{1\}$ Nullmengen sind, gilt
\begin{align*}
 \int_{(-1,1)} |f| \df m 
 &= 2 \lim_{t \to 0^+} \int_{t}^{1} x^{-1} \df x \\
 &= 2 \lim_{t \to 0^+} (\ln(x))|_t^{1} \\
 &= \infty.
\end{align*}
Somit gilt $f \notin L_1((-1,1))$.

%%%%%%%%%%%%%%%%%%%%%%%%%%%%%%%%%%%%%%%%%%%%%%%%%%%%%%%%%%%%%
\section*{Aufgabe 3}
Es seien $X = [0,1]$ und
$m$ das Lebesgue-Mass auf $X$.
Beweise, dass die Folge $(f_n)_{n \in \N}$
definiert durch
\[
  f_n := \chi_{\left[\frac{j}{2^k}, \frac{j+1}{2^k}\right]},
\]
$n = 2^k + j, j \in \{0, 1, \dots, 2^k - 1\}, k \in \N$,
im Mass konvergiert, aber für kein $x \in X$ die Folge
$(f_n(x))_{n \in \N}$ konvergiert.

\begin{proof}
\uline{Für kein $x \in X$ konvergiert $(f_n(x))_{n \in \N}$:}
Seien $x \in X, k \in \N_{\geq 2}$.
Dann existieren $i,j \in \{0,1,\dots,2^k-1\}$,
sodass $x \in [i/2^k, (i+1)/2^k]$, $x \notin [j/2^k, (j+1)/2^k]$.
Also hat die Folge $(f_n(x))_{n \in \N}$ zwei Häufungspunkte (0 und 1).
Also konvergiert sie nicht.

\uline{$(f_n)_{n \in \N}$ konvergiert im Mass:}
Da $f_n(x) > 0$ genau dann, wenn ein $j \in \{0, 1, \dots, 2^k-1\}$
existiert mit $x \in [j/2^k, (j+1)/2^k]$, gilt
\begin{align*}
  \lim_{n \to \infty} m(\{x \in [0,1] : f_n(x) > 0\})
  &= \lim_{n \to \infty} m(\left[\frac{j}{2^k},\frac{j+1}{2^k}\right]) \\
  &= \lim_{n \to \infty} \frac{1}{2^k} \\
  &= 2 \lim_{n \to \infty} \frac{1}{2^{k+1}}.
\end{align*}
Da $n = 2^k + j \leq 2^{k+1}$:
\begin{align*}
  0 \leq \lim_{n \to \infty} \frac{1}{2^{k+1}}
  \leq \lim_{n \to \infty} \frac{1}{2^{n}}
  = 0.
\end{align*}
Folglich konvergiert $(f_n)_{n \in \N}$ im Mass, und zwar gegen die Nullfunktion.
\end{proof}

%%%%%%%%%%%%%%%%%%%%%%%%%%%%%%%%%%%%%%%%%%%%%%%%%%%%%%%%%%%%%
\section*{Aufgabe 4}
Es seien $\mathfrak{M}$ die $\sigma$-Algebra
der Lebesgue-messbaren Mengen in $\R$ und $m$
das Lebesgue-Mass. Zeige, dass zu jedem
$A \in \mathfrak{M}$ mit $m(A) > 0$ ein
$B \in \mathfrak{M}$ existiert mit
\[
  B \subset A, ~~ m(A \setminus B) > 0, ~~ m(B) > 0.
\]
Man sagt daher ``Das Lebesgue-Mass hat keine Atome.''
\begin{proof}
Wegen Satz 5.3 existieren abgeschlossene $A_i \subset A$
mit $m(\bigcup_{i = 1}^{\infty} A_i) = m(A)$.
Für mindestens ein $i \in \N$ gilt $0 < m(A_i) =: c$.
Definiere nun für alle $z \in \Z$
\[
  B_z = \left[\frac{z}{2}c, \frac{z+1}{2}c\right].
\]
Dann $m(B_z) = c/2$.
Die $B_z$ sind abgeschlossen, also Borelmengen.
Also $A_i \cap B_z \in \mathfrak{B} \subset \mathfrak{M}$ für alle $z \in Z$.
Da $A_i = A_i \cap \bigcup_{z \in \Z} B_z$,
müssen $z, z' \in \Z, z \neq z',$ bestehen mit
\[
  c/2 \geq m(A_i \cap B_z) > 0,~~ c/2 \geq m(A_i \cap B_{z'}) > 0.
\]
Definiere $B := A_i \cap B_z$. Dann $B \subset A_i \subset A$,
$0 < m(B) < m(A)/2$. Folglich $m(A\setminus B) > m(A)/2$.
\end{proof}

\end{document}
