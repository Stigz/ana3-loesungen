\documentclass[10pt]{article}\usepackage[]{graphicx}\usepackage[]{color}
% maxwidth is the original width if it is less than linewidth
% otherwise use linewidth (to make sure the graphics do not exceed the margin)
\makeatletter
\def\maxwidth{ %
  \ifdim\Gin@nat@width>\linewidth
    \linewidth
  \else
    \Gin@nat@width
  \fi
}
\makeatother

\definecolor{fgcolor}{rgb}{0.345, 0.345, 0.345}
\newcommand{\hlnum}[1]{\textcolor[rgb]{0.686,0.059,0.569}{#1}}%
\newcommand{\hlstr}[1]{\textcolor[rgb]{0.192,0.494,0.8}{#1}}%
\newcommand{\hlcom}[1]{\textcolor[rgb]{0.678,0.584,0.686}{\textit{#1}}}%
\newcommand{\hlopt}[1]{\textcolor[rgb]{0,0,0}{#1}}%
\newcommand{\hlstd}[1]{\textcolor[rgb]{0.345,0.345,0.345}{#1}}%
\newcommand{\hlkwa}[1]{\textcolor[rgb]{0.161,0.373,0.58}{\textbf{#1}}}%
\newcommand{\hlkwb}[1]{\textcolor[rgb]{0.69,0.353,0.396}{#1}}%
\newcommand{\hlkwc}[1]{\textcolor[rgb]{0.333,0.667,0.333}{#1}}%
\newcommand{\hlkwd}[1]{\textcolor[rgb]{0.737,0.353,0.396}{\textbf{#1}}}%
\let\hlipl\hlkwb

\usepackage{framed}
\makeatletter
\newenvironment{kframe}{%
 \def\at@end@of@kframe{}%
 \ifinner\ifhmode%
  \def\at@end@of@kframe{\end{minipage}}%
  \begin{minipage}{\columnwidth}%
 \fi\fi%
 \def\FrameCommand##1{\hskip\@totalleftmargin \hskip-\fboxsep
 \colorbox{shadecolor}{##1}\hskip-\fboxsep
     % There is no \\@totalrightmargin, so:
     \hskip-\linewidth \hskip-\@totalleftmargin \hskip\columnwidth}%
 \MakeFramed {\advance\hsize-\width
   \@totalleftmargin\z@ \linewidth\hsize
   \@setminipage}}%
 {\par\unskip\endMakeFramed%
 \at@end@of@kframe}
\makeatother

\definecolor{shadecolor}{rgb}{.97, .97, .97}
\definecolor{messagecolor}{rgb}{0, 0, 0}
\definecolor{warningcolor}{rgb}{1, 0, 1}
\definecolor{errorcolor}{rgb}{1, 0, 0}
\newenvironment{knitrout}{}{} % an empty environment to be redefined in TeX

\usepackage{alltt}

\title{Serie 6}
\date{}
\usepackage{setspace}
\setstretch{1}
\usepackage{parskip}

\usepackage[utf8]{inputenc}
\usepackage[T1]{fontenc}
\usepackage{ngerman}
\usepackage{hyphenat}
\usepackage{natbib}
\usepackage[normalem]{ulem}

\usepackage[a4paper, top=2cm, bottom=2cm, left=2cm, right=2cm]{geometry}

% \usepackage[small, explicit]{titlesec}
% \usepackage[normalem]{ulem}
% \titleformat{\paragraph}[runin]{\normalfont\normalsize\itshape}{\theparagraph}{1em}{#1}
% \titleformat{\subparagraph}[runin]{\normalfont\normalsize}{\thesubparagraph}{1em}{\uline{#1}}
% \titlespacing*{\subparagraph}{0pt}{1ex}{1em}


% \usepackage{marginnote}
% \renewcommand*{\marginfont}{\footnotesize}

\usepackage{tikz}
\usetikzlibrary{calc, intersections, arrows}

\usepackage{amsmath}
\usepackage{amsfonts}
\usepackage{amssymb}
\usepackage{amsthm}
\usepackage{mathtools}
\usepackage{mathrsfs}

% \usepackage{tgpagella}
% \usepackage[sc]{mathpazo}

\renewcommand*{\proofname}{Beweis}
\newtheorem*{lemma}{Lemma}


\author{}
% \date{\today}

% \usepackage{booktabs}
% \usepackage{fancyhdr}
% \usepackage{lastpage}
% \pagestyle{fancy}
% \fancyhf{}
% \makeatletter
% \let\runauthor\@author
% \let\runtitle\@title
% \makeatother
% \lhead{\runauthor}
% \rhead{S.~\thepage~von \pageref{LastPage}\\\today}
% \chead{\runtitle}

\usepackage[shortlabels]{enumitem}

% \usepackage{microtype}
             
\newcommand{\N}{\mathbb{N}}
\newcommand{\Z}{\mathbb{Z}}
\newcommand{\Q}{\mathbb{Q}}
\newcommand{\R}{\mathbb{R}}
\newcommand{\K}{\mathbb{K}}
\newcommand{\C}{\mathbb{C}}
\newcommand{\df}{\,\textrm{d}}
\newcommand{\pr}{\,\textrm{pr}}

% \newcommand*{\QED}[1][$\square$]{%
% \leavevmode\unskip\penalty9999 \hbox{}\nobreak\hfill
%     \quad\hbox{#1}%
% }
                                
\hyphenation{Äqui-valenz-re-la-tion Äqui-va-lenz-klasse Prim-faktor-zer-legung 
Prim-faktor-zer-legungen unter-schied-lich-en sur-jek-tiv Po-tenzen 
unter-schied-liche In-duk-tions-an-nahme Differential-gleichung}

\usepackage{hyperref}
\IfFileExists{upquote.sty}{\usepackage{upquote}}{}
\begin{document}

%%%%%%%%%%%%%%%%%%%%%%%%%%%%%%%%%%%%%%%%%%%%%%%%%%%%%%%%%%%%%%%%%%%%%%%%%%%%%%%%
\maketitle

%%%%%%%%%%%%%%%%%%%%%%%%%%%%%%%%%%%%%%%%%%%%%%%%%%%%%%%%%%%%%%%%%%%
%%%%%%%%%%%%%%%%%%%%%%%%%%%%%%%%%%%%%%%%%%%%%%%%%%%%%%%%%%%%%%%%%%%

%%%%%%%%%%%%%%%%%%%%%%%%%%%%%%%%%%%%%%%%%%%%%%%%%%%%%%%%%%%%%%%%%%%
\section*{Aufgabe 1}
Zeige, dass das Achsenkreuz in $\R^2$ Lebesgue-Mass Null hat.
\begin{proof} 
Für das Achsenkreuz $A \subset \R^2$ gilt
\begin{align*}
  A 
  = \bigcup_{n = 0}^{\infty} \left( \underbrace{\{(x, 0) : n \leq x < n + 1\}}_{=: V_n} \cup \underbrace{\{(x, 0) : -n \leq x < -n + 1\}}_{=: V_{-n}} \cup \underbrace{\{(0, y) : n \leq y < n + 1\}}_{=: W_n} \cup  \underbrace{\{(0, y) : -n \leq y < -n + 1\}}_{=: W_{-n}}\right ).
\end{align*}

$V_n$ ist eine 2-Zelle mit $a_1 = n, b_1 = n + 1, a_2 = b_2 = 0$;
$W_n$ ist eine 2-Zelle mit $a_2 = n, b_2 = n + 1, a_1 = b_1 = 0$; $n \in \N$.
Also $\textrm{vol}~V_n = \textrm{vol}~W_n = 0$.
Analog: $\textrm{vol}~V_{-n} = \textrm{vol}~W_{-n} = 0$.
Folglich $\textrm{vol}(V_n \cup V_{-n} \cup  W_n \cup W_{-n}) = 0$.
Folglich $m(A) = \textrm{vol}(\bigcup_{n = 0}^{\infty} (V_n \cup V_{-n} \cup  W_n \cup W_{-n}) = 0$.
\end{proof}

%%%%%%%%%%%%%%%%%%%%%%%%%%%%%%%%%%%%%%%%%%%%%%%%%%%%%%%%%%%%%%%%%%%
\section*{Aufgabe 2}
Es sei $m$ das Lebesgue-Mass auf $\R$.
Zeige:
\begin{enumerate}[(a)]
  \item Die Funktion
  \begin{align*}
    f: \R &\to \R,\\
    x &\mapsto
    \begin{cases}
      \frac{\sin(x)}{x}, & x \neq 0, \\
      1, & x = 0,
    \end{cases}
  \end{align*}
  ist stetig, messbar und Riemann-integrierbar, aber nicht Lebesgue-integrierbar bzgl.\ $m$,
  und das Lebesgue-Integral $\int_{\R} f \df m$ ist nicht definiert.
  \begin{proof}
    \uline{$f$ ist stetig:} Klar für $x \neq 0$. Für $x = 0$, verwende L'Hôpital:
    \[
      \lim_{x \to 0} \frac{\sin(x)}{x} = \lim_{x \to 0} \cos(x) = 1 = f(0).
    \]
    Also $f$ stetig bei $x = 0$.

    \uline{$f$ ist messbar:} $U \subset \R$ offen, dann $f^{-1}(U)$ offen,
    da $f$ stetig ist. Offene Mengen sind Borelmengen. Also ist $f$ messbar.

    \uline{$f$ ist Riemann-integrierbar:}
    Für $t \in \R$ finden wir durch partielle Integration:
    \begin{align*}
    \int_{\pi/2}^t f(x) \df x
    &= \left[-\frac{\cos(x)}{x}\right]^t_{\pi/2} - \int_{\pi/2}^t \frac{\cos(x)}{x^2} \df x \\
    &= -\frac{\cos(t)}{t} - \int_{\pi/2}^t \frac{\cos(x)}{x^2} \df x.
    \end{align*}
    Da $\cos(x) \in [-1,1]$ gilt einerseits
    \begin{align*}
      \int_{\pi/2}^t f(x) \df x
      &\geq -\frac{\cos(t)}{t} - \int_{\pi/2}^t \frac{1}{x^2} \df x \\
      &= -\frac{\cos(t)}{t} + \frac{1}{t} - \frac{2}{\pi}
    \end{align*}
    und andererseits
    \begin{align*}
       \int_{\pi/2}^t f(x) \df x  &\leq -\frac{\cos(t)}{t} - \frac{1}{t} + \frac{2}{\pi}.
    \end{align*}
    Folglich:
    \[
      \left|\lim_{t \to \infty} \int_{\pi/2}^t f(x) \df x\right| \leq \frac{2}{\pi} < \infty.
    \]
    Weiter $|\int_0^{\pi/2} f(x) \df x| < \infty$, da $f$ stetig ist.
    Da
    \[\int_{-\infty}^{\infty}f(x) \df x = 2\left(\int_0^{\pi/2} f(x) \df x + \int_{\pi/2}^{\infty} f(x) \df x\right) < \infty,\]
    ist $f$ Riemann-integrierbar.

    \uline{$f$ ist nicht Lebesgue-integrierbar bzgl.\ $m$:}
    Für stetige Funktionen auf einem kompakten Intervall
    stimmen das Rieman- und das Lebesgue-Integral überein:
    $\int_{[0, t]} |f| \df m = \int_0^t |f(x)| \df x$.
    Berechne für $k \in \N_{\geq 1}$:
    \begin{align*}
    \int_0^{k\pi} \left|\frac{\sin(x)}{x}\right| \df x
    &= \int_0^{\pi} \left|\frac{\sin(x)}{x}\right| \df x +
       \int_{\pi}^{2\pi} \left|\frac{\sin(x)}{x}\right| \df x +
       \dots +
       \int_{(k-1)\pi}^{k\pi} \left|\frac{\sin(x)}{x}\right| \df x \\
    &\geq \frac{1}{\pi}\int_0^{\pi} |\sin(x)| \df x +
       \frac{1}{2\pi}\int_{\pi}^{2\pi} |\sin(x)| \df x +
       \dots +
       \frac{1}{k\pi}\int_{(k-1)\pi}^{k\pi} |\sin(x)| \df x \\
    &\geq \frac{1}{\pi}\int_0^{\pi} \sin(x) \df x +
       \frac{1}{2\pi}\int_{\pi}^{2\pi} \sin(x) \df x +
       \dots +
       \frac{1}{k\pi}\int_{(k-1)\pi}^{k\pi} \sin(x) \df x.
    \end{align*}
    Da $\int_{n\pi}^{(n+1)\pi} \sin(x) \df x = \int_{0}^{\pi} \sin(x) \df x = 2$ für alle $n \in \N$:
    \[
     \int_0^{k\pi} \left|\frac{\sin(x)}{x}\right| \df x \geq \frac{2}{\pi}\sum_{i=1}^n\frac{1}{i}.
    \]
    Für $k \to \infty$ erhalten wir eine Reihe von
    Integralen, die durch die harmonische Reihe minorisiert wird
    und folglich divergiert. Folglich $f \notin L_1(\R)$.
    Folglich ist das Lebesgue-Integral von $f$ über $\R$ bzgl.\ $m$
    nicht definiert (siehe Def.\ 3.1).
  \end{proof}

\item Die charakteristische Funktion $\chi_{\Q}$ von $\Q \subset \R$
ist Lebesgue-integrierbar bzgl.\ $m$, aber nicht Riemann-integrierbar.
Bestimme ausserdem $\int_{\R} \chi_{\Q} \df m$.
  \begin{proof}
  \uline{$\chi_{\Q}$ ist nicht Riemann-integrierbar:}
  Jedes Intervall von $[0,1]$ enthält sowohl rationale als irrationale
  Zahlen. Also
  \[
    \uline{S}(\chi_{\Q}, P) = 0,~~
    \overline{S}(\chi_{\Q}, P) = 1
  \]
  für jede Partition $P$ von $[0,1]$.
  Also ist $\chi_{\Q}$ nicht einmal auf $[0,1]$
  Riemann-integrierbar.

  \uline{$\chi_{\Q}$ ist Lebesgue-integrierbar; $\int_{\R} \chi_{\Q} \df m$:}
  $\Q$ ist eine abzählbare Vereinigung von Punkten.
  Punkte in $\R$ haben Lebesgue-Mass 0 (1-Zelle mit $a = b$).
  Also ist $\Q$ eine Nullmenge bzgl. $m$.
  Nullmengen sind beim Lebesgue-Integral unerheblich.
  Also $\int_{\R} |\chi_{\Q}| \df m = \int_{\R} 0 \df m = 0 < \infty$. Also $\chi_{\Q} \in L_1(\R)$.
  \end{proof}
\end{enumerate}

%%%%%%%%%%%%%%%%%%%%%%%%%%%%%%%%%%%%%%%%%%%%%%%%%%%%%%%%%%%%%%%%%%%
\section*{Aufgabe 3}
Die Cantor-Menge in $\R$ ist definiert als
\[
  T := \left\{\sum_{n=1}^{\infty} \frac{x_n}{3^n} : x_n \in \{0, 2\}\right\}.
\]
Beweise:
\begin{enumerate}[(a)]
  \item $T = [0,1] \setminus \bigcup_{n=1}^{\infty} M_n$, wobei
  \[
    M_n := \left\{\sum_{n=1}^{\infty} \frac{x_n}{3^n} : x_j \in \{0, 2\}, j < n, x_n = 1, (x_j)_{j=n+1}^{\infty} \neq \{0\}, \{2\}\right\}, n \in \N
  \]
  ist.
  \begin{proof}
   $t \in T \Rightarrow t \in [0,1]$ ist klar.
   Falls $m \in M_n$ für irgendein $n \in \N$, dann ist in der Ternärdarstellung von $m$
   mindestens eine $1$ notwendig. $T$ besteht jedoch gerade aus Zahlen ohne 1 in (einer möglichen)
   Ternärdarstellung. Also $m \notin T$.
  \end{proof}

  \item $T$ ist abgeschlossen.
  \begin{proof}
  Jedes $M_n$ ist eine Vereinigung von $2^{n-1}$ offenen Intervallen
  (je mit Länge $1/3^n$) und ist folglich offen.
  Also ist auch $\bigcup_{n = 1}^{\infty}$ offen.
  Wegen Teil (a) ist $T^c$ offen in $\R$. Also ist $T$ abgeschlossen.
  \end{proof}

  \item $T$ ist überabzählbar. Zeige dazu, dass die Abbildung
  \begin{align*}
    \varphi : T &\to [0,1],\\
    \sum_{n=1}^{\infty} \frac{x_n}{3^n}  &\mapsto \sum_{n=1}^{\infty} \frac{y_n}{2^n}, y_j := \frac{x_j}{2}, j \in \N,
  \end{align*}
  wohldefiniert und surjektiv ist.
  \begin{proof}
  \uline{Die Abbildung $\varphi$ ist wohldefiniert:}
  Die Ternärdarstellung von $t \in T$, mit $x_n \in \{0, 2\}$, ist eindeutig.
  Grund: Angenommen, $t \in T$ hätte zwei unterschiedliche Ternärdarstellungen
  mit $x_n,y_n \in \{0,2\}$ für alle $n \in \N$. Dann gibt es eine erste Stelle $k$,
  wo sich diese unterscheiden.
  Also würde gelten
  \[
  \frac{0}{3^k} + \sum_{n = k+1}^{\infty} \frac{x_n}{3^n} = \frac{2}{3^k} + \sum_{n = k+1}^{\infty} \frac{y_n}{3^n}
  \]
  mit $x_n, y_n \in \{0,2\}$ für alle $n \geq k+1$.
  Das ist äquivalent zu
  \[
    2 = \sum_{n = k+1}^{\infty} \frac{x_n-y_n}{3^{k+n}}.
  \]
  Wegen $x_n-y_n \leq 2$ gilt aber
  \[
  \sum_{n = k+1}^{\infty} \frac{x_n-y_n}{3^{k+n}} \leq 2 \sum_{n = 1}^{\infty} \frac{1}{3^n} = 2\left(\frac{1}{1-1/3}-1\right) = 1.
  \]
  Widerspruch.

  Folglich ist $\varphi(t)$ für $t \in T$ eindeutig.
  Für jedes $t \in T$ gilt:
  \[
  0 = \sum_{i = 1}^{\infty}\frac{0}{2^i} \leq \varphi(t) \leq \sum_{i = 1}^{\infty}\frac{1}{2^i} = 1.
  \]
  Also ist $\varphi$ wohldefiniert.

  \uline{$\varphi(T) = [0,1]$:}
  Alle Zahlen im Intervall $[0,1]$
  haben genau eine Binärdarstellung, die als regulärer Ausdruck $(0.(0|1)^*)_2$
  geschrieben werden kann,
  d.h., $[0,1] \ni x = \sum_{i=1}^{\infty}\frac{b_i}{2^i}$,
  $b_i \in \{0,1\}, i \geq 1$.
  Für jede solche Binärdarstellung gibt es
  ein $t = \sum_{i=1}^{\infty}\frac{2b_i}{3^i}$
  mit $2b_i \in \{0, 2\}$, also $t \in T$.
  Also $\varphi(T) = [0,1]$.
  Da $[0,1]$ überabzählbar ist, ist $T$ dies auch.
  \end{proof}
\end{enumerate}

%%%%%%%%%%%%%%%%%%%%%%%%%%%%%%%%%%%%%%%%%%%%%%%%%%%%%%%%%%%%%%%%%%%
\section*{Aufgabe 4}
Berechne $m(T)$ für die Cantor-Menge $T$ aus Aufgabe 3
und das Lebesgue-Mass $m$ auf $\R$.

\paragraph{Lösung.}
Aus Aufgabe 3:
$\textrm{vol}~M_n = \frac{1}{2} \left(\frac{2}{3}\right)^n$.
Da $M_k \cap M_{\ell} = \emptyset, k \neq \ell$, gilt
\[
  \textrm{vol}\left(\bigcup_{i = 1}^{\infty} M_i\right) = \frac{1}{2}\sum_{i = 1}^{\infty} \left(\frac{2}{3}\right)^i = \frac{1}{2} \left(\frac{1}{1-2/3} - 1\right) = 1.
\]
Wegen $T = [0,1] \setminus \bigcup_{i = 1}^{\infty} M_i$, folgt
$m(T) = \textrm{vol}~T = 1 - 1 = 0$.

\end{document}
