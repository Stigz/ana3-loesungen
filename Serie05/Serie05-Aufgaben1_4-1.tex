\documentclass[10pt]{article}\usepackage[]{graphicx}\usepackage[]{color}
% maxwidth is the original width if it is less than linewidth
% otherwise use linewidth (to make sure the graphics do not exceed the margin)
\makeatletter
\def\maxwidth{ %
  \ifdim\Gin@nat@width>\linewidth
    \linewidth
  \else
    \Gin@nat@width
  \fi
}
\makeatother

\definecolor{fgcolor}{rgb}{0.345, 0.345, 0.345}
\newcommand{\hlnum}[1]{\textcolor[rgb]{0.686,0.059,0.569}{#1}}%
\newcommand{\hlstr}[1]{\textcolor[rgb]{0.192,0.494,0.8}{#1}}%
\newcommand{\hlcom}[1]{\textcolor[rgb]{0.678,0.584,0.686}{\textit{#1}}}%
\newcommand{\hlopt}[1]{\textcolor[rgb]{0,0,0}{#1}}%
\newcommand{\hlstd}[1]{\textcolor[rgb]{0.345,0.345,0.345}{#1}}%
\newcommand{\hlkwa}[1]{\textcolor[rgb]{0.161,0.373,0.58}{\textbf{#1}}}%
\newcommand{\hlkwb}[1]{\textcolor[rgb]{0.69,0.353,0.396}{#1}}%
\newcommand{\hlkwc}[1]{\textcolor[rgb]{0.333,0.667,0.333}{#1}}%
\newcommand{\hlkwd}[1]{\textcolor[rgb]{0.737,0.353,0.396}{\textbf{#1}}}%
\let\hlipl\hlkwb

\usepackage{framed}
\makeatletter
\newenvironment{kframe}{%
 \def\at@end@of@kframe{}%
 \ifinner\ifhmode%
  \def\at@end@of@kframe{\end{minipage}}%
  \begin{minipage}{\columnwidth}%
 \fi\fi%
 \def\FrameCommand##1{\hskip\@totalleftmargin \hskip-\fboxsep
 \colorbox{shadecolor}{##1}\hskip-\fboxsep
     % There is no \\@totalrightmargin, so:
     \hskip-\linewidth \hskip-\@totalleftmargin \hskip\columnwidth}%
 \MakeFramed {\advance\hsize-\width
   \@totalleftmargin\z@ \linewidth\hsize
   \@setminipage}}%
 {\par\unskip\endMakeFramed%
 \at@end@of@kframe}
\makeatother

\definecolor{shadecolor}{rgb}{.97, .97, .97}
\definecolor{messagecolor}{rgb}{0, 0, 0}
\definecolor{warningcolor}{rgb}{1, 0, 1}
\definecolor{errorcolor}{rgb}{1, 0, 0}
\newenvironment{knitrout}{}{} % an empty environment to be redefined in TeX

\usepackage{alltt}

\title{Series 5}
\date{}
\usepackage{setspace}
\setstretch{1}
\usepackage{parskip}

\usepackage[utf8]{inputenc}
\usepackage[T1]{fontenc}
\usepackage{ngerman}
\usepackage{hyphenat}
\usepackage{natbib}

\usepackage[a4paper, top=2cm, bottom=2cm, left=2cm, right=2cm]{geometry}

% \usepackage[small, explicit]{titlesec}
% \usepackage[normalem]{ulem}
% \titleformat{\paragraph}[runin]{\normalfont\normalsize\itshape}{\theparagraph}{1em}{#1}
% \titleformat{\subparagraph}[runin]{\normalfont\normalsize}{\thesubparagraph}{1em}{\uline{#1}}
% \titlespacing*{\subparagraph}{0pt}{1ex}{1em}


% \usepackage{marginnote}
% \renewcommand*{\marginfont}{\footnotesize}

\usepackage{tikz}
\usetikzlibrary{calc, intersections, arrows}

\usepackage{amsmath}
\usepackage{amsfonts}
\usepackage{amssymb}
\usepackage{amsthm}
\usepackage{mathtools}
\usepackage{mathrsfs}

% \usepackage{tgpagella}
% \usepackage[sc]{mathpazo}

\renewcommand*{\proofname}{Beweis}
\newtheorem*{lemma}{Lemma}


% \author{Jan Vanhove\\10-219-186}
% \date{\today}

% \usepackage{booktabs}
% \usepackage{fancyhdr}
% \usepackage{lastpage}
% \pagestyle{fancy}
% \fancyhf{}
% \makeatletter
% \let\runauthor\@author
% \let\runtitle\@title
% \makeatother
% \lhead{\runauthor}
% \rhead{S.~\thepage~von \pageref{LastPage}\\\today}
% \chead{\runtitle}

\usepackage[shortlabels]{enumitem}

% \usepackage{microtype}
             
\newcommand{\N}{\mathbb{N}}
\newcommand{\Z}{\mathbb{Z}}
\newcommand{\Q}{\mathbb{Q}}
\newcommand{\R}{\mathbb{R}}
\newcommand{\K}{\mathbb{K}}
\newcommand{\C}{\mathbb{C}}
\newcommand{\df}{\,\textrm{d}}
\newcommand{\pr}{\,\textrm{pr}}

% \newcommand*{\QED}[1][$\square$]{%
% \leavevmode\unskip\penalty9999 \hbox{}\nobreak\hfill
%     \quad\hbox{#1}%
% }
                                
\hyphenation{Äqui-valenz-re-la-tion Äqui-va-lenz-klasse Prim-faktor-zer-legung 
Prim-faktor-zer-legungen unter-schied-lich-en sur-jek-tiv Po-tenzen 
unter-schied-liche In-duk-tions-an-nahme Differential-gleichung}

\usepackage{hyperref}
\IfFileExists{upquote.sty}{\usepackage{upquote}}{}
\begin{document}

%%%%%%%%%%%%%%%%%%%%%%%%%%%%%%%%%%%%%%%%%%%%%%%%%%%%%%%%%%%%%%%%%%%%%%%%%%%%%%%%
\maketitle

%%%%%%%%%%%%%%%%%%%%%%%%%%%%%%%%%%%%%%%%%%%%%%%%%%%%%%%%%%%%%%%%%%%
%%%%%%%%%%%%%%%%%%%%%%%%%%%%%%%%%%%%%%%%%%%%%%%%%%%%%%%%%%%%%%%%%%%
\section*{Exercise 1}
Check if the condition
\[
  \int_X |f(x)| \df x < \infty
\]
ist met for the Riemann integral for the following functions $f: X \to \R$:
\begin{enumerate}[(i)]
  \item $X = (1, \infty)$, $f(x) := x^s, s \in \R$.

  \textit{Answer.}  For $s = -1$:
  \[
    \int_X |x^s| \df x = \lim_{x \to \infty} \ln(x) - \ln(1) = \infty.
  \]


  For $s \neq 1$:
  \[
    \int_X |x^s| \df x = \frac{1}{s+1} (\lim_{x \to \infty} x^{s+1} - 1) = \begin{cases}
      \infty, & s > -1,\\
      -\frac{1}{s+1}, & s < -1.
    \end{cases}.
  \]
  So `no' for $s \geq -1$ and `yes' for $s < -1$.

  \item $X = \R, f(x) := e^{-x^2}$.

  \textit{Answer.}
  This'll probably be the wrong way round, but here goes:
  The probability density function
  of the standard normal is given by $g: \R \to \R$:
  \[
    g(x) = \frac{1}{\sqrt{2\pi}}e^{-x^2/2}.
  \]
  So $g(x) = \sqrt{\frac{f(x)}{2\pi}}$.
  So $f(x) = 2\pi (g(x))^2$.
  Since $0 < g(\R) < 1$,
  $(g(x))^2 < g(x)$ for all $x \in \R$.
  Since $\int_{-\infty}^{\infty} g(x) \df x = 1$,
  $\int_{-\infty}^{\infty} (g(x))^2 \df x \leq 1$.
  So $\int_{-\infty}^{\infty} f(x) \df x \leq 2\pi < \infty$.
  So `yes'.
\end{enumerate}

\section*{Exercise 2}
\begin{enumerate}[(a)]
  \item Let $(X, \tau)$ be a topological space.
  Show that $C_c(X)$ is a vector subspace
  of $C(X)$. Further show that $f, g \in C_c(X) \Rightarrow fg \in C_c(X)$.
  \begin{proof}
  $C_c(X)$ is a vector subspace:
  \begin{itemize}
    \item $0 \in C(X)$ is continuous. $\textrm{supp~} 0 = \overline{\emptyset} = \emptyset$, which is compact.
    So $0 \in C_c(X)$.

    \item $f, g \in C_c(X)$. Then $f + g$ is continuous.
    If $f(x) = g(x) = 0$, then $(f+g)(x) = 0$.
    Further, $\overline{A \cup B} = \overline{A} \cup \overline{B}$.
    Hence:
    \begin{align*}
      \textrm{supp~} (f+g)
      &= \overline{\{x \in X : f(x) + g(x) \neq 0\}} \\
      &\subset \overline{\{x \in X : f(x) \neq 0 \lor g(x) \neq 0\}} \\
      &= \overline{\{x \in X : f(x) \neq 0\} \cup \{x \in X :g(x) \neq 0\}} \\
      &= \textrm{supp~} f \cup \textrm{supp~} g,
    \end{align*}
    which is compact as the union of two compact sets. Closed subsets of 
    compact sets are compact, too, hence $ \textrm{supp~} (f+g)$ is compact.
    So $f+g \in C_c(X)$.

    \item $f \in C_c(X), \alpha \in \C$. Then $\alpha f$ is continuous.
    If $\alpha = 0, \alpha f \equiv 0 \in C_c(X)$.
    If $\alpha \neq 0$, $\alpha f(x) \neq 0 \Leftrightarrow f(x) \neq 0$. Hence
    $\textrm{supp~} \alpha f = \textrm{supp~} f$, which is compact. So $\alpha f \in C_c(X)$.
  \end{itemize}

  For $f,g \in C_c(X)$, $fg$ is continuous.
  Further: $(fg)(x) \neq 0 \Leftrightarrow f(x) \neq 0 \land g(x) \neq 0$.
  So
  \[
    \textrm{supp~} fg  = \overline{\{x \in X: f(x) \neq 0\} \cap \{x \in X : g(x) \neq 0\}} \subset \textrm{supp~} f.
  \]
  Closed subsets of compact sets are compact. So $fg \in C_c(X)$.
  \end{proof}

  \item Let $f : \R \to \R, f:= \chi_{[-\pi, \pi]}$ and
  \begin{align*}
    g: \R &\to \R,\\
    x &\mapsto
    \begin{cases}
      \sin(x),    & x \geq 0,\\
      0,          & x < 0.
    \end{cases}
  \end{align*}
    Determine $\textrm{supp~} f, \textrm{supp~} g, \textrm{supp~} fg$.
    Check if $f,g,fg \in C_c(\R)$.

    Further: For $a \in \R, \varepsilon > 0$,
    define $f_{(\varepsilon, a)}: \R \to \R, f_{(\varepsilon, a)}(x) := f(\varepsilon x - a)$.
    Determine $\textrm{supp~} f_{(\varepsilon, a)}$.

    \textit{Answer.} $f \notin C_c(\R)$ since it's not continuous.
    \[
      \textrm{supp~} f = \overline{\{x \in \R: f(x) \neq 0\}} = \overline{[-\pi, \pi]} = [-\pi, \pi].
    \]

    $g \notin C_c(\R)$ since its support isn't bounded.
        \[
      \textrm{supp~} g = \overline{\{x \in \R: g(x) \neq 0\}} = \overline{\R_{\geq 0} \setminus \pi \Z} = \R_{\geq 0}.
    \]

    $fg \in C_c(\R)$ since it's continuous 
    ($(fg)|_{[0, \pi]} \equiv \sin(x)|_{[0, \pi]}, (fg)|_{[0, \pi]^c} \equiv 0|_{[0, \pi]^c}$, 
    and $\sin(0) = \sin(\pi) = 0$) 
    and its support is compact.
    \begin{align*}
      \textrm{supp~} fg
      &= \overline{\{x \in \R : f(x)g(x) \neq 0\}} \\
      &= \overline{\{x \in [-\pi, \pi] : g(x) \neq 0\}} \\
      &= \overline{\{x \in [0, \pi] : \sin(x) \neq 0\}} \\
      &= \overline{(0, \pi)} = [0, \pi].
    \end{align*}

    Further:
    \begin{align*}
      \textrm{supp~} f_{(\varepsilon, a)}
      &= \overline{\{x \in \R : f(\varepsilon x - a) \neq 0\}} \\
      &= \overline{\{x \in \R : \varepsilon x - a \in [-\pi, \pi]\}} \\
      &= \overline{\{x \in \R : x \in [\frac{a - \pi}{\varepsilon}, \frac{a + \pi}{\varepsilon}]\}} \\
      &= [\frac{a - \pi}{\varepsilon}, \frac{a + \pi}{\varepsilon}].
    \end{align*}
\end{enumerate}

%%%%%%%%%%%%%%%%%%%%%%%%%%%%%%%%%%%%%%%%%%%%%%%%%%%%%%%%%%%%%%%%%

\section*{Exercise 3}
\begin{enumerate}[(a)]
  \item Let the hypotheses of Riesz representation theorem be met. Prove the following:
  \begin{itemize}
    \item $E \in \mathfrak{M}$ is $\sigma$-compact $\Rightarrow$ $E$ has $\sigma$-finite measure.
    \begin{proof}
     Since $E$ is $\sigma$-compact, there exist compact $K_i \subset X, i \in \N,$ such that 
     $E = \bigcup_{i = 1}^{\infty} K_i$.
     By Theorem 4.7(i), $\mu(K_i) < \infty$ for all $i \in \N$.
    \end{proof}
    
    \item $E \in \mathfrak{M}$ has $\sigma$-finite measure $\Rightarrow$ $E$ is inner regular.
    \begin{proof}
      For $E$ with $\mu(E) < \infty$, Theorem 4.7 suffices.
      For $\mu(E) = \infty$, we may write $E = \bigcup_{i = 1}^{\infty} E_i$,
      for $E_i$ with $\mu(E_i) < \infty$ for all $i \in \N$. Note that
      \[
        \mu(\bigcup_{i = 1}^n E_i) \leq \sum_{i = 1}^n \mu(E_i) < \infty.
      \]
      Hence, by Theorem 4.7,
      \[
        \mu(\bigcup_{i = 1}^n E_i) = \sup\{\mu(K) : K \subset \bigcup_{i = 1}^n E_i, \textrm{$K$ compakt}\}.
      \]
      As $n \to \infty$, we obtain
      \[
        \mu(\bigcup_{i = 1}^\infty E_i) = \mu(E) = \sup\{\mu(K) : K \subset E, \textrm{$K$ compakt}\}.\qedhere
      \]
    \end{proof}
  \end{itemize}
  
  \item Let $(X, d)$ be a metric space, $\lambda$ a Borel measure on $X$
  with $\lambda(K) < \infty$ for all compact $K \subset X$. Prove that
  \begin{align*}
    \Lambda: C_c(X, \R) &\to \R,\\
    f &\mapsto \int_X f \df \lambda
  \end{align*}
  is a monotonous positive linear form.
  \begin{proof}
  \begin{enumerate}[(1)]
    \item $\Lambda$ is well-defined:\\
    $f \in C_c(X, \R)$ implies $f(X) \subset \R$ compact (comment re: Definition 4.3), i.e., bounded
    by a constant function $c(x) = \beta \in \R, x \in X$.
    $f \in C_c(X, \R)$ also implies that $f$ is non-zero on a (subset of)
    a compact set $K \subset X$. So $\int_X f \df \lambda \leq \beta \mu(K) < \infty$.
    
    \item $\Lambda$ is a map from a $\C$ vector space to $\R \subset \C$.
    
    \item $\Lambda$ is $\C$-linear:\\ 
    From (1): $f,g \in C_c(X, \R) \Rightarrow f,g \in \mathscr{L}_1(\lambda)$.
    $\Lambda(f + \alpha g) = \Lambda(f) + \alpha\Lambda(g)$ follows from Theorem 3.2.
    
    \item $\Lambda$ is positive since by Proposition 2.6:
    \[
      0 \leq f \Rightarrow 0 = \int_X 0 \df \lambda \leq \int_X f \df \lambda.
    \]
  \end{enumerate}

    So $\Lambda$ is a positive linear form.
    All positive linear forms are monotonous (as proved in lecture).
  \end{proof}
\end{enumerate}

\section*{Exercise 4 (Lebesgue sums)}
Let $(X, \mathfrak{M}, \mu)$ be a measure space and
let $f : X \to \R$ be positive, bounded and measurable.
Define
\[
  A_j := \{x \in X : \alpha + \frac{(j - 1)(\beta - \alpha)}{n} \leq f(x) < \alpha + \frac{j(\beta - \alpha)}{n}\}
\]
and
\[
  A_n := \{x \in X : \alpha + \frac{(n-1)(\beta - \alpha)}{n} \leq f(x) \leq \beta\}
\]
for $j = 1, \dots, n-1, n \in \N$ and with
$\alpha := \inf\{f(x) : x \in X\}$,
$\beta := \sup\{f(x): x \in X\}$.
Prove that
\[
  \int_X f \df \mu = \lim_{n \to \infty} s_n,
\]
where
\[
  s_n := \sum_{j = 1}^n \left(\alpha + \frac{(j-1)(\beta - \alpha)}{n}\right) \mu(A_j),
\]
$n \in \N$.
\begin{proof}
  % If $\int_X f \df \mu = \infty$, then, because of the boundedness, $\mu(X) = 
  Note that $s_n = \int_X z_n \df \mu$, where
  $z_n : X \to \R, z_n := \sum_{j = 1}^n \left(\alpha + \frac{(j-1)(\beta - \alpha)}{n}\right) \chi_{A_j}, n \in \N,$
  is a simple measurable function.
  
  Further note that $z_n \leq f$: For $i \neq j, A_i \cap A_j = \emptyset$.
  Hence, $z_n(x) = \alpha + \frac{(k-1)(\beta - \alpha)}{n} \leq f(x)$ for one $k \in \{1, \dots, n\}$.
  Since $f$ is bounded, so is $z_n$.
  
  Moreover, $z_n \to f, n \to \infty$:
  As $n \to \infty$, $\left(\alpha + \frac{j(\beta - \alpha)}{n}\right) - \left(\alpha + \frac{(j - 1)(\beta - \alpha)}{n}\right) \to 0$
  for all $j \in \{0, \dots, j-n\}$. Similarly,
  $\beta - \left(\alpha + \frac{(n-1)(\beta - \alpha)}{n}\right) \to 0$.
  So for $n \to \infty$ and for $k = 1, \dots, n$, $z_n(x) = \frac{(k-1)(\beta - \alpha)}{n} \to f(x)$.
  
  Now assume that there exists a $g \in \mathscr{L}_1$ such that $g \geq z_n$ for all $n$.
  Then by dominated convergence, we obtain
  \[
    \lim_{n \to \infty} s_n = \lim_{n \to \infty} \int_X z_n \df \mu = \int_X \lim_{n \to \infty} z_n \df \mu = \int_X f \df \mu. 
  \]
  If no such $g$ exists, then $f \notin \mathscr{L}_1$, so $\int_X f \df \mu = \infty$. By Fatou:
  \[
    \infty = \int_X f \df \mu = \int_X \lim_{n \to \infty} z_n \df \mu \leq \liminf_{n \to \infty} \int_X z_n \df \mu = \liminf_{n \to \infty} s_n,
  \]
  so $s_n \to \infty, n \to \infty$.
\end{proof}
\end{document}
