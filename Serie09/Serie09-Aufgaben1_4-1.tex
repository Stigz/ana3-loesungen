\documentclass[10pt]{article}\usepackage[]{graphicx}\usepackage[]{color}
% maxwidth is the original width if it is less than linewidth
% otherwise use linewidth (to make sure the graphics do not exceed the margin)
\makeatletter
\def\maxwidth{ %
  \ifdim\Gin@nat@width>\linewidth
    \linewidth
  \else
    \Gin@nat@width
  \fi
}
\makeatother

\definecolor{fgcolor}{rgb}{0.345, 0.345, 0.345}
\newcommand{\hlnum}[1]{\textcolor[rgb]{0.686,0.059,0.569}{#1}}%
\newcommand{\hlstr}[1]{\textcolor[rgb]{0.192,0.494,0.8}{#1}}%
\newcommand{\hlcom}[1]{\textcolor[rgb]{0.678,0.584,0.686}{\textit{#1}}}%
\newcommand{\hlopt}[1]{\textcolor[rgb]{0,0,0}{#1}}%
\newcommand{\hlstd}[1]{\textcolor[rgb]{0.345,0.345,0.345}{#1}}%
\newcommand{\hlkwa}[1]{\textcolor[rgb]{0.161,0.373,0.58}{\textbf{#1}}}%
\newcommand{\hlkwb}[1]{\textcolor[rgb]{0.69,0.353,0.396}{#1}}%
\newcommand{\hlkwc}[1]{\textcolor[rgb]{0.333,0.667,0.333}{#1}}%
\newcommand{\hlkwd}[1]{\textcolor[rgb]{0.737,0.353,0.396}{\textbf{#1}}}%
\let\hlipl\hlkwb

\usepackage{framed}
\makeatletter
\newenvironment{kframe}{%
 \def\at@end@of@kframe{}%
 \ifinner\ifhmode%
  \def\at@end@of@kframe{\end{minipage}}%
  \begin{minipage}{\columnwidth}%
 \fi\fi%
 \def\FrameCommand##1{\hskip\@totalleftmargin \hskip-\fboxsep
 \colorbox{shadecolor}{##1}\hskip-\fboxsep
     % There is no \\@totalrightmargin, so:
     \hskip-\linewidth \hskip-\@totalleftmargin \hskip\columnwidth}%
 \MakeFramed {\advance\hsize-\width
   \@totalleftmargin\z@ \linewidth\hsize
   \@setminipage}}%
 {\par\unskip\endMakeFramed%
 \at@end@of@kframe}
\makeatother

\definecolor{shadecolor}{rgb}{.97, .97, .97}
\definecolor{messagecolor}{rgb}{0, 0, 0}
\definecolor{warningcolor}{rgb}{1, 0, 1}
\definecolor{errorcolor}{rgb}{1, 0, 0}
\newenvironment{knitrout}{}{} % an empty environment to be redefined in TeX

\usepackage{alltt}

\title{Serie 9}
\date{}
\usepackage{setspace}
\setstretch{1}
\usepackage{parskip}

\usepackage[utf8]{inputenc}
\usepackage[T1]{fontenc}
\usepackage{ngerman}
\usepackage{hyphenat}
\usepackage{natbib}
\usepackage[normalem]{ulem}

\usepackage[a4paper, top=2cm, bottom=2cm, left=2cm, right=2cm]{geometry}

% \usepackage[small, explicit]{titlesec}
% \usepackage[normalem]{ulem}
% \titleformat{\paragraph}[runin]{\normalfont\normalsize\itshape}{\theparagraph}{1em}{#1}
% \titleformat{\subparagraph}[runin]{\normalfont\normalsize}{\thesubparagraph}{1em}{\uline{#1}}
% \titlespacing*{\subparagraph}{0pt}{1ex}{1em}


% \usepackage{marginnote}
% \renewcommand*{\marginfont}{\footnotesize}

\usepackage{tikz}
\usetikzlibrary{calc, intersections, arrows}

\usepackage{amsmath}
\usepackage{amsfonts}
\usepackage{amssymb}
\usepackage{amsthm}
\usepackage{mathtools}
\usepackage{mathrsfs}

% \usepackage{tgpagella}
% \usepackage[sc]{mathpazo}

\renewcommand*{\proofname}{Beweis}
\newtheorem*{lemma}{Lemma}


\author{}
% \date{\today}

% \usepackage{booktabs}
% \usepackage{fancyhdr}
% \usepackage{lastpage}
% \pagestyle{fancy}
% \fancyhf{}
% \makeatletter
% \let\runauthor\@author
% \let\runtitle\@title
% \makeatother
% \lhead{\runauthor}
% \rhead{S.~\thepage~von \pageref{LastPage}\\\today}
% \chead{\runtitle}

\usepackage[shortlabels]{enumitem}

% \usepackage{microtype}
             
\newcommand{\N}{\mathbb{N}}
\newcommand{\Z}{\mathbb{Z}}
\newcommand{\Q}{\mathbb{Q}}
\newcommand{\R}{\mathbb{R}}
\newcommand{\K}{\mathbb{K}}
\newcommand{\C}{\mathbb{C}}
\newcommand{\df}{\,\textrm{d}}
\newcommand{\pr}{\,\textrm{pr}}
\DeclareMathOperator*{\esssup}{ess\,sup}

% \newcommand*{\QED}[1][$\square$]{%
% \leavevmode\unskip\penalty9999 \hbox{}\nobreak\hfill
%     \quad\hbox{#1}%
% }
                                
\hyphenation{Äqui-valenz-re-la-tion Äqui-va-lenz-klasse Prim-faktor-zer-legung 
Prim-faktor-zer-legungen unter-schied-lich-en sur-jek-tiv Po-tenzen 
unter-schied-liche In-duk-tions-an-nahme Differential-gleichung}

\usepackage{hyperref}
\IfFileExists{upquote.sty}{\usepackage{upquote}}{}
\begin{document}

%%%%%%%%%%%%%%%%%%%%%%%%%%%%%%%%%%%%%%%%%%%%%%%%%%%%%%%%%%%%%%%%%%%%%%%%%%%%%%%%
\maketitle

%%%%%%%%%%%%%%%%%%%%%%%%%%%%%%%%%%%%%%%%%%%%%%%%%%%%%%%%%%%%%%%%%%%
\section*{Aufgabe 1}
Es seien $f, g: \R^n \to \R$ stetig und $\R^n$ mit dem Lebesgue-Mass
versehen. Zeige:
\begin{enumerate}[(a)]
  \item Gilt $f = g$ fast überall, so ist $f = g$ überall.
        
        \begin{proof}
        Insbesondere $f - g$ ist stetig und fast überall gleich
        der Nullabbildung. Definiere
        \[
          A := (f-g)^{-1}((-\infty, 0)) \cup (f-g)^{-1}((0, \infty)).
        \]
        Da $f - g \equiv 0$ fast überall, gilt $m(A) = 0$.
        Andererseits ist $A$ offen, da stetige Urbilder offener Mengen
        dies sind. Die einzige offene Menge in $\R^n$ mit Mass 0
        ist die Nullmenge: Sobald wir einen offenen Ball mit
        positivem Radius in $A$ zeichnen können, hat $A$ positives Mass.
        Also ist $f-g \equiv 0$ überall.
        \end{proof}
        
  \item $\esssup f = \sup f.$
        
        \begin{proof}
          Definiere
          \[
            M := \{a \in \R : f^{-1}((a, \infty)) = \emptyset \}
          \]
          und stelle fest, dass
          \[
            \sup f = 
            \begin{cases}
              \inf M, & M \neq \emptyset,\\
              \infty, & M = \emptyset.
            \end{cases}
          \]
          Definiere $S$ wie in Definition 6.4.
          Ist $f$ stetig, so gilt mit der gleichen Logik wie 
          unter 1(a), dass $m(f^{-1}((a, \infty))) = 0$
          genau dann, wenn $f^{-1}((a, \infty)) = \emptyset$.
          Also $M = S$ und damit die Behauptung.
        \end{proof}
\end{enumerate}

%%%%%%%%%%%%%%%%%%%%%%%%%%%%%%%%%%%%%%%%%%%%%%%%%%%%%%%%%%%%%%%%%%%
\section*{Aufgabe 2}
Es sei $p \in [1, \infty]$. Für welche $r,s \in \R$ liegen folgende
Funktionen in $\mathscr{L}_p(\R_+)$:
\begin{enumerate}[(a)]
  \item $f: \R_+ \to \R, f(x) = x^r$.
  
        {\bf Antwort.} $f$ ist klar stetig und positiv auf seinem Definitionsbereich.
        
        \uline{Fall 1: $p = \infty$:} Mit Aufgabe 1(b) gilt
        \[
          \|x^r\|_{\infty} := \esssup|x^r| = \sup x^r.
        \]
        Für $r > 0$, gilt $\lim_{x \to \infty} x^r = \infty$, also $\sup x^r = \infty$.
        Für $r < 0$, gilt $\lim_{x \to 0}x^r = \infty$, also $\sup x^r = \infty$.
        Für $r = 0$, gilt $\sup x^r = \sup x^0 = 1$. Also
        \[
          x^r \in \mathscr{L}_{\infty}(\R_+) \Leftrightarrow r = 0.
        \]
        
        \uline{Fall 2: $p \in [1, \infty)$:}
        Es reicht, zu überprüfen, ob $\int_{\R_+}x^{rp} \df m < \infty$.
        
        Für $rp = -1$ (d.h, $r = -1/p$) gilt
        \[
        \int_{\R_+} x^{-1} \df m = \ln(x)|^{\infty}_0 = \infty.
        \]
        
        Für $rp > -1$ (d.h., $r > -1/p$):
        \[
          \int_{\R_+} x^{rp} \df m = \frac{1}{rp+1} x^{rp+1}|_0^{\infty} = \underbrace{\frac{1}{rp+1}}_{> 0} \lim_{x \to \infty} x^{rp+1} = \infty.
        \]

        Für $rp < -1$ (d.h., $r < -1/p$):
        \[
          \int_{\R_+} x^{rp} \df m = \frac{1}{rp+1} x^{rp+1}|_0^{\infty} = \underbrace{\frac{1}{rp+1}}_{< 0} \left(\underbrace{\lim_{x \to \infty} x^{rp+1}}_{= 0} - \underbrace{\lim_{x \to 0} x^{rp+1}}_{= \infty} \right) = \infty.
        \]
        
        Also gilt für alle $p \neq \infty$, dass $x^r \notin \mathscr{L}_{p}(\R_+)$.
  
  \item $g: \R_+ \to \R, g(x) = \begin{cases}x^r, & 0 < x \leq 1,\\ x^s, & x > 1. \end{cases}$
  
        {\bf Antwort.} 
        \uline{Fall 1: $p = \infty$:}
        $1^r = 1 = 1^s$ für alle $s,r \in \R$, also ist $g$ stetig.
        Damit $\esssup g = \sup g < \infty$, muss $r \geq 0, s \leq 0$.
        
        \uline{Fall 2: $p \in [1, \infty)$:}
        Für $rp = -1$ gilt 
        \[
          \int_0^1 x^{-1} \df x = \lim_{t \to 0} \left(\ln(x)|^{1}_t\right) = \infty.
        \]
        Für $rp < -1$:
        \[
          \int_0^1 x^{rp} \df x = \frac{1}{rp+1} \lim_{t \to 0}\left((x^{rp+1})|^1_t\right) = \infty.
        \]
        Für $rp > -1$:
        \[
          \int_0^1 x^{rp} \df x = \frac{1}{rp+1} \left(x^{rp+1}\right)|^1_0 = \frac{1}{rp+1} < \infty.
        \]
        Also $\int_0^{1} x^{rp} \df x < \infty$
        genau dann, wenn $rp > -1$, also für $r > -1/p$.
        
        Für $sp = -1$ gilt
        \[
          \int_1^{\infty} x^{-1} \df x = \lim_{t \to \infty} \left(\ln(x)|^{t}_1\right) = \infty.
        \]
        Für $sp > -1$:
        \[
          \int_1^{\infty} x^{sp} \df x = \frac{1}{sp+1} \lim_{t \to \infty}\left((x^{sp+1})|^t_1\right) = \infty.
        \]
        Für $sp < -1$:
        \[
          \int_1^{\infty} x^{sp} \df x = \frac{1}{sp+1} \lim_{t \to \infty}\left((x^{sp+1})|^t_1\right) = - \frac{1}{sp+1} < \infty.
        \]
        $\int_1^{\infty} x^{sp} \df x < \infty$
        genau dann, wenn $sp < -1$, also für $s < -1/p$.
        
        Insgesamt $g \in \mathscr{L}_p(\R_+)$ genau dann,
        wenn $r > -1/p, s < -1/p$.
        
\end{enumerate}

%%%%%%%%%%%%%%%%%%%%%%%%%%%%%%%%%%%%%%%%%%%%%%%%%%%%%%%%%%%%%%%%%%%
\section*{Aufgabe 3}
Es seien $0 < p_1 < p_2 < \infty$. Beweise:
\begin{enumerate}[(a)]
  \item Ist $(X, \mathfrak{M}, \mu)$ ein Massraum
        mit $\mu(X) < \infty$, so gilt
        $\mathscr{L}_{p_2}(\mu) \subset \mathscr{L}_{p_1}(\mu)$.
        
        \begin{proof}
        Es gelte $f \in \mathscr{L}_{p_2}(\mu)$
        Da $p_1 < p_2$, gilt $\frac{p_1}{p_2} + \frac{1}{r} = 1$ mit $r = \frac{p_2}{p_2-p_1} > 1$.
        Mit Hölder folgt
          \begin{align*}
            \int_X |f|^{p_1} \df \mu
            &= \int_X 1 |f|^{p_1} \df \mu \\
            &\leq \left(\int_X 1^r \df \mu\right)^{1/r} \cdot \left(\int_X |f|^{p_1p_2/p_1} \df \mu \right)^{p_1/p_2} \\
            &= (\mu(X))^{1/r} \cdot \left(\left(\int_X |f|^{p_2} \df \mu \right)^{1/p_2}\right)^{p_1}.
          \end{align*}
          Wegen $f \in \mathscr{L}_{p_2}(\mu)$ 
          gilt $\left(\int_X |f|^{p_2} \df \mu \right)^{1/p_2} \in \R$,
          also $\left(\left(\int_X |f|^{p_2} \df \mu \right)^{1/p_2}\right)^{p_1} \in \R$.
          Nach Voraussetzung auch $\mu(X) < \infty$.
          Insgesamt:
          \[
             \int_X |f|^{p_1} \df \mu < \infty.
          \]
          % Daraus
          % \[
          %    \left(\int_X |f|^{p_1} \df \mu\right)^{1/p_1} < \infty.
          % \]
          Also $f \in \mathscr{L}_{p_1}$.
        \end{proof}
        
  \item Ist $X$ eine abzählbare (endliche oder unendliche)
        Menge und $\mu$ das Zählmass auf $X$,
        so gilt $\mathscr{L}_{p_1}(\mu) \subset \mathscr{L}_{p_2}(\mu)$.
        Die Inklusion ist strikt, falls $X$ 
        abzählbar unendlich ist.
        
        \begin{proof}
          Wenn $f$ bzgl.\ des Zählmasses $\mu$ 
          Lebesgue-integrierbar ist, gilt
          \[
            \int_X |f|^r \df \mu = \sum_{x \in X} (|f(x)|)^r,
          \]
          $r \in \R$.
          Für $X$ endlich haben wir es mit einer endlichen,
          also konvergenten, Summe zu tun, vorausgesetzt,
          dass die Funktion nur endliche Werte annimmt.
          Daher: 
          \[
            \sum_{x \in X} (|f(x)|)^{p_1} \df \mu < \infty 
            \Rightarrow |f(X)| < \infty 
            \Rightarrow \sum_{x \in X} (|f(x)|)^{p_2} \df \mu < \infty.
          \]
          
          Im unendlichen Fall heisst
          \[
            \sum_{k=1}^{\infty} |f(x_k)|^{p_1} < \infty,
          \]
          dass diese Reihe absolut konvergiert. Also
          dürfen wir die Summanden so umordnen, dass
          $|f(x_k)|^{p_1} \geq |f(x_{k+1})|^{p_1}$ für alle $k \in \N$.
          Wegen der Konvergenz muss weiter ein $N \in \N$
          existieren, sodass $|f(x_k)|^{p_1} < 1$
          für alle $k \geq N$. Da $p_2 > p_1$, gilt für solche
          $k$ dann
          $|f(x_k)|^{p_2} \leq  |f(x_k)|^{p_1}$. Also
          \begin{align*}
            \sum_{k=1}^{\infty} |f(x_k)|^{p_2}
            &= \sum_{k=1}^{N-1} |f(x_k)|^{p_2} + \sum_{k=N}^{\infty} |f(x_k)|^{p_2}.
          \end{align*}
          Der erste Teil ist endlich, also konvergent;
          der zweite Teil wird von der konvergenten
          Reihe $\sum_{k=N}^{\infty} |f(x_k)|^{p_1}$
          majorisiert und ist also auch konvergent.
          Somit 
          $\mathscr{L}_{p_1}(\mu) \subset \mathscr{L}_{p_2}(\mu)$.
          
          Betrachten wir jedoch zu vorgegebenem $p_1$
          und $X = \{x_1, x_2, \dots\}$
          die Abbildung
          \begin{align*}
            f:  X     &\to       \C,\\
                x_k   &\mapsto   k^{-1/p_1},
          \end{align*}
          so stellen wir fest, dass für $p_2 > p_1$ gilt:
          \[
            \sum_{k=1}^{\infty} |f(x_k)|^{p_2} = \sum_{k=1}^{\infty} \frac{1}{k^{p_2/p_1}} < \infty,
          \]
          da $p_2/p_1 > 1$ (allgemeine harmonische Reihe).
          Jedoch
          \[
            \sum_{k=1}^{\infty} |f(x_k)|^{p_1} = \sum_{k=1}^{\infty} \frac{1}{k},
          \]
          was gegen $\infty$ divergiert.
          Also ist die obige Inklusion strikt.
        \end{proof}
        
  \item Es sei $X = [0, \infty)$ versehen mit dem Lebesgue-Mass.
        Es gilt sowohl $\mathscr{L}_{p_1}(X) \setminus \mathscr{L}_{p_2}(X) \neq \emptyset$ als auch
        $\mathscr{L}_{p_2}(X) \setminus \mathscr{L}_{p_1}(X) \neq \emptyset$.
        
        \begin{proof}
          Seien $0 < p_1 < p_2 < \infty$ vorgegeben.
          Definiere die fallende Treppenfunktion $f: X \to \R$ durch
          $f(x) = k^{-p_1}$ für $x \in [k-1, k), k = 1, 2, \dots$.
          \footnote{Gemeint ist: $f(x) = \sum_{k=1}^{\infty} \frac{1}{k^{p_1}} \chi_{[k-1,k)}$.}
          Dann 
          \[
            \int_X (f)^{p_1} \df m = 1 + \frac{1}{2} + \frac{1}{3} + \dots = \infty.
          \]
          Aber
          \[
            \int_X (f)^{p_2} \df m = 1 + \frac{1}{2^r} + \frac{1}{3^r} + \dots < \infty,
          \]
          da $r := p_2-p_1 > 1$ (allgemeine harmonische Reihe).
          Also $f \in \mathscr{L}_{p_2}(X) \setminus \mathscr{L}_{p_1}(X)$.
          
          Definiere nun 
          \begin{align*}
            g: X  &\to      \R,\\
               x  &\mapsto  
                  \begin{cases}
                    x^{-1/p_2},    & x \in (0, 1),\\
                    0,            & \textrm{sonst}.
                  \end{cases}
          \end{align*}
          Dann 
          \[
            \int_0^{\infty} (g(x))^{p_2} \df x = \int_0^{1} 1/x \df x = \infty,
          \]
          aber
          \[
            \int_0^{\infty} (g(x))^{p_1} \df x = \int_0^{1} x^{-p_1/p_2} \df x = \frac{p_2}{p_2-p_1} < \infty.
          \]
           Also $g \in \mathscr{L}_{p_1}(X) \setminus \mathscr{L}_{p_2}(X)$.
        \end{proof}
\end{enumerate}

%%%%%%%%%%%%%%%%%%%%%%%%%%%%%%%%%%%%%%%%%%%%%%%%%%%%
\section*{Aufgabe 4}
Es seien $a < b$ und $f: [a,b] \to \R$ Lebesgue-integrierbar.
Zeige, dass die Funktion
\begin{align*}
  F: [a,b] &\to \R, \\
  x &\mapsto \int_{[a,x]} f \df m,
\end{align*}
absolutstetig ist.

\begin{proof}
Sei $\varepsilon > 0$ vorgegeben.
Da $|f| \in \mathscr{L}_1([a,b])$ und $|f| \geq 0$,
existiert mit Aufgabe 3.4 ein $\delta > 0$, sodass für alle $A \in \mathfrak{M}$
mit $m(A) < \delta$ gilt, dass $\int_A |f| \df m < \varepsilon$.
Sei nun 
$a \leq \alpha_1 < \beta_1 \leq \alpha_2 < \beta_2 \leq \dots \leq a_n < \beta_n \leq b$
eine Partition von $[a,b]$ mit der Eigenschaft, dass
$A := \bigsqcup_{j = 1}^n [\alpha_j, \beta_j]$, $m(A) < \delta$.
Das heisst nichts anderes als $\sum_{j = 1}^{n} \beta_j - \alpha_j < \delta$.
Man stellt nun fest:
\begin{align*}
  \sum_{j = 1}^n \left|\int_{[a, \beta_j]}f \df m - \int_{[a, \alpha_j]} f \df m\right|
  &= \sum_{j = 1}^n \left|\int_{[\alpha_j, \beta_j]}f \df m\right| \\
  &\leq \sum_{j = 1}^n \int_{[\alpha_j, \beta_j]}|f| \df m\\
  &= \int_A |f| \df m < \varepsilon.\qedhere
\end{align*}
\end{proof}
\end{document}
