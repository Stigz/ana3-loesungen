\documentclass[10pt]{article}\usepackage[]{graphicx}\usepackage[]{color}
% maxwidth is the original width if it is less than linewidth
% otherwise use linewidth (to make sure the graphics do not exceed the margin)
\makeatletter
\def\maxwidth{ %
  \ifdim\Gin@nat@width>\linewidth
    \linewidth
  \else
    \Gin@nat@width
  \fi
}
\makeatother

\definecolor{fgcolor}{rgb}{0.345, 0.345, 0.345}
\newcommand{\hlnum}[1]{\textcolor[rgb]{0.686,0.059,0.569}{#1}}%
\newcommand{\hlstr}[1]{\textcolor[rgb]{0.192,0.494,0.8}{#1}}%
\newcommand{\hlcom}[1]{\textcolor[rgb]{0.678,0.584,0.686}{\textit{#1}}}%
\newcommand{\hlopt}[1]{\textcolor[rgb]{0,0,0}{#1}}%
\newcommand{\hlstd}[1]{\textcolor[rgb]{0.345,0.345,0.345}{#1}}%
\newcommand{\hlkwa}[1]{\textcolor[rgb]{0.161,0.373,0.58}{\textbf{#1}}}%
\newcommand{\hlkwb}[1]{\textcolor[rgb]{0.69,0.353,0.396}{#1}}%
\newcommand{\hlkwc}[1]{\textcolor[rgb]{0.333,0.667,0.333}{#1}}%
\newcommand{\hlkwd}[1]{\textcolor[rgb]{0.737,0.353,0.396}{\textbf{#1}}}%
\let\hlipl\hlkwb

\usepackage{framed}
\makeatletter
\newenvironment{kframe}{%
 \def\at@end@of@kframe{}%
 \ifinner\ifhmode%
  \def\at@end@of@kframe{\end{minipage}}%
  \begin{minipage}{\columnwidth}%
 \fi\fi%
 \def\FrameCommand##1{\hskip\@totalleftmargin \hskip-\fboxsep
 \colorbox{shadecolor}{##1}\hskip-\fboxsep
     % There is no \\@totalrightmargin, so:
     \hskip-\linewidth \hskip-\@totalleftmargin \hskip\columnwidth}%
 \MakeFramed {\advance\hsize-\width
   \@totalleftmargin\z@ \linewidth\hsize
   \@setminipage}}%
 {\par\unskip\endMakeFramed%
 \at@end@of@kframe}
\makeatother

\definecolor{shadecolor}{rgb}{.97, .97, .97}
\definecolor{messagecolor}{rgb}{0, 0, 0}
\definecolor{warningcolor}{rgb}{1, 0, 1}
\definecolor{errorcolor}{rgb}{1, 0, 0}
\newenvironment{knitrout}{}{} % an empty environment to be redefined in TeX

\usepackage{alltt}

\title{Serie 4, Aufgabe 2}
\date{}
\usepackage{setspace}
\setstretch{1}
\usepackage{parskip}

\usepackage[utf8]{inputenc}
\usepackage[T1]{fontenc}
% \usepackage{ngerman}
\usepackage{hyphenat}
\usepackage{natbib}

\usepackage[a4paper, top=2cm, bottom=2cm, left=2cm, right=2cm]{geometry}

% \usepackage[small, explicit]{titlesec}
% \usepackage[normalem]{ulem}
% \titleformat{\paragraph}[runin]{\normalfont\normalsize\itshape}{\theparagraph}{1em}{#1}
% \titleformat{\subparagraph}[runin]{\normalfont\normalsize}{\thesubparagraph}{1em}{\uline{#1}}
% \titlespacing*{\subparagraph}{0pt}{1ex}{1em}


% \usepackage{marginnote}
% \renewcommand*{\marginfont}{\footnotesize}

\usepackage{tikz}
\usetikzlibrary{calc, intersections, arrows}

\usepackage{amsmath}
\usepackage{amsfonts}
\usepackage{amssymb}
\usepackage{amsthm}
\usepackage{mathtools}
\usepackage{mathrsfs}

% \usepackage{tgpagella}
% \usepackage[sc]{mathpazo}

\renewcommand*{\proofname}{Proof}
\newtheorem*{lemma}{Lemma}


% \author{Jan Vanhove\\10-219-186}
% \date{\today}

% \usepackage{booktabs}
% \usepackage{fancyhdr}
% \usepackage{lastpage}
% \pagestyle{fancy}
% \fancyhf{}
% \makeatletter
% \let\runauthor\@author
% \let\runtitle\@title
% \makeatother
% \lhead{\runauthor}
% \rhead{S.~\thepage~von \pageref{LastPage}\\\today}
% \chead{\runtitle}

\usepackage[shortlabels]{enumitem}

% \usepackage{microtype}
             
\newcommand{\N}{\mathbb{N}}
\newcommand{\Z}{\mathbb{Z}}
\newcommand{\Q}{\mathbb{Q}}
\newcommand{\R}{\mathbb{R}}
\newcommand{\K}{\mathbb{K}}
\newcommand{\C}{\mathbb{C}}
\newcommand{\df}{\,\textrm{d}}
\newcommand{\pr}{\,\textrm{pr}}

% \newcommand*{\QED}[1][$\square$]{%
% \leavevmode\unskip\penalty9999 \hbox{}\nobreak\hfill
%     \quad\hbox{#1}%
% }
                                
\hyphenation{Äqui-valenz-re-la-tion Äqui-va-lenz-klasse Prim-faktor-zer-legung 
Prim-faktor-zer-legungen unter-schied-lich-en sur-jek-tiv Po-tenzen 
unter-schied-liche In-duk-tions-an-nahme Differential-gleichung}

\usepackage{hyperref}
\IfFileExists{upquote.sty}{\usepackage{upquote}}{}
\begin{document}

%%%%%%%%%%%%%%%%%%%%%%%%%%%%%%%%%%%%%%%%%%%%%%%%%%%%%%%%%%%%%%%%%%%%%%%%%%%%%%%%
\maketitle

%%%%%%%%%%%%%%%%%%%%%%%%%%%%%%%%%%%%%%%%%%%%%%%%%%%%%%%%%%%%%%%%%%%
%%%%%%%%%%%%%%%%%%%%%%%%%%%%%%%%%%%%%%%%%%%%%%%%%%%%%%%%%%%%%%%%%%%
Let $(X, \mathfrak{M}, \mu)$ be a measure space
and $f : X \to Y$ a map.
Define
\[
  \mathfrak{N} := \{A \subset Y : f^{-1}(A) \in \mathfrak{M}\},~~~
  \nu(A) := \mu(f^{-1}(A)), A \in \mathfrak{N}.
\]
Show that $(Y, \mathfrak{N}, \nu)$ is a measure space.
Find all $\mathfrak{N}$-measurable maps $g: Y \to \C$
and show that 
\[
  \int_Y g \df \nu = \int_X g \circ f \df \mu,
\]
if either side exists.

\begin{proof}
$\mathfrak{N}$ is a $\sigma$-algebra (Def.~1.3):
\begin{enumerate}[(i)]
  \item $f^{-1}(Y) = X \in \mathfrak{M} \Rightarrow Y \in \mathfrak{N}$.

  \item $A \in \mathfrak{N} \Rightarrow f^{-1}(A) \in \mathfrak{M} \Rightarrow  (f^{-1}(A))^c = f^{-1}(A^c) \in \mathfrak{M} \Rightarrow A^C \in \mathfrak{N}$.

  \item $A_i \in \mathfrak{N}, i \in \N \Rightarrow f^{-1}(A_i) \in \mathfrak{M}, i \in \N \Rightarrow \bigcup_{n = 1}^{\infty} f^{-1}(A_i) = f^{-1}(\bigcup_{n = 1}^{\infty} A_i) \in \mathfrak{M} \Rightarrow \bigcup_{n = 1}^{\infty} A_i \in \mathfrak{N}$.
\end{enumerate}

$\nu$ is a measure on $\mathfrak{N}$ (Def.~2.1):
\begin{enumerate}[(i)]
  \item Let $A_1, A_2, \dots \in \mathfrak{N}$ be disjoint.
  Then $f^{-1}(A_1), f^{-1}(A_2), \dots \in \mathfrak{M}$ are
  also disjoint. Hence:
  \[
    \nu \left(\bigcup_{i = 1}^{\infty}A_i\right)
    = \mu \left(f^{-1}\left(\bigcup_{i = 1}^{\infty}A_i\right)\right)
    = \mu\left(\bigcup_{i = 1}^{\infty} f^{-1}(A_i)\right) 
    = \sum_{i=1}^{\infty}\mu(f^{-1}(A_i)) 
    = \sum_{i=1}^{\infty}\nu(A_i).
  \]
  
  \item $f^{-1}(\emptyset_Y) = \emptyset_X \in \mathfrak{M}.$ $\mu(\emptyset_X) = 0$. Hence $\nu(\emptyset_Y) = 0 < \infty$.
\end{enumerate}

Let $V \in \C$ be open. Then
\begin{align*}
  \textrm{$g : Y \to \C$ is measurable}
  &\Leftrightarrow g^{-1}(V) \in \mathfrak{N} \\
  &\Leftrightarrow f^{-1}(g^{-1})(V) = (g\circ f)^{-1}(V) \in \mathfrak{M}.
\end{align*}
So $g : Y \to \C$ is measurable iff $g \circ f : X \to \C$ is measurable.

If $\int_Y g \df \nu$ exists, then
\[
  \int_Y g \df \nu = 
  \int_Y (\textrm{Re}~ g)^+ \df \nu
  -  \int_Y (\textrm{Re}~ g)^- \df \nu
  + \textrm{i}\left(\int_Y (\textrm{Im}~ g)^+ \df \nu 
  -  \int_Y (\textrm{Im}~ g)^- \df \nu\right).
\]
Now consider a simple measurable $s$ with 
$0 \leq s \leq \textrm{Re}(g)^+$,
$s = \sum_{i = 1}^n \alpha_i \chi_{A_i}$,
$\alpha_i \in \R^+, A_i \in \mathfrak{N}$.
Then $0 \leq s \circ f \leq \textrm{Re}(g \circ f)^+$
is also simple and measurable.
Since $\chi_{f^{-1}(A)} = \chi_A \circ f$:
\begin{align*}
  \int_Y s \df \nu
  = \int_{Y} \sum_{i = 1}^n \alpha_i \nu(A_i)
  &= \sum_{i = 1}^n \alpha_i \mu(f^{-1}(A_i)) \\
  &= \int_{f^{-1}(Y)} \sum_{i = 1}^n \alpha_i \chi_{f^{-1}(A_i)} \df \mu \\
  &= \int_{X} \sum_{i = 1}^n \alpha_i \chi_{A_i} \circ f \df \mu \\
  &= \int_{X} s \circ f \df \mu.
\end{align*}
From this, $\int_Y (\textrm{Re}~ g)^+ \df \nu \leq \int_X (\textrm{Re}~ g \circ f)^+ \df \mu$.
Since $Y \supset f(X)$, `$\geq$' holds, too.
Hence $\int_Y (\textrm{Re}~ g)^+ \df \nu = \int_X (\textrm{Re}~ g \circ f)^+ \df \mu$.
Proceed analogously for the other terms that make up $\int_Y g \df \nu$.
\end{proof}
\end{document}
